% !TEX root=../main.tex

\section{Dynamic TopHat}
\label{sec:dyntophat}

\subsection{Language}

Although it is possible in \TOPHAT\ to start multiple tasks in parallel,
the amount of these task should be specified by the programmer at develop time.

\begin{example}[Vending machine with shared stock]
  \label{exm:vending-shared}
  Take this slightly more complex vending machine,
  where two buyers can purchase a $\Snack$ at the same time from a shared inventory.

  \begin{TASK}[]
    let vend : Task Snack = do ss.
      enter Int >>= do n.
      case n of
        1 |-> decrease Biscuit ss >>= do _.
             view Biscuit
        2 |-> decrease Chocolate ss >>= do _.
             view Chocolate
        _ |-> fail
    let machine : Task (Snack * Snack) =
      share [(Biscuit, 24), (Chocolate, 16)] >>= do ss.
      vend ss <&> vend ss
  \end{TASK}

  Here \var{decrease} is a function that decreases the amount of the given snack by one in the shared list of snacks.
  Note that this task results in a pair containing the two bought $\Snack$s.
\end{example}

\DYNTOPHAT\ is a moderate extension to the \TOPHAT\ base language where we add support for \emph{task pools}.
A task pool is a new combinator at the task level of the \TOPHAT\ language.
The combinator ($\Pool^k_{t_0} []$) is parametrised by an unique key $k$, a template task $t_0$ and (initially) an empty task list $[]$.
Each time end users send an \emph{init} action to this pool ($k\Send\Init$),
the template task is placed in the list of current running tasks,
in this case resulting in $\Pool^k_{t_0} [t_0]$.
This way, an arbitrary amount of tasks can be started at runtime.
Tasks in the task list of a pool can also be killed by sending a \emph{kill} action to the pool ($k\Send\Kill1$).
In this case the first task in the pool will be killed.
The syntactic extensions to \TOPHAT\ are shown in \cref{fig:dynamic-grammar}.

\begin{example}[Dynamic number of vending machines]
  \label{exm:vending-dynamic}
  Using the same \var{vend} task from \cref{exm:vending-shared},
  we can now create a vending machine where end users can start an arbitrary amount of vending tasks.
  \begin{TASK}
    let machine : Task (List Snack) =
      share [(Biscuit, 24), (Chocolate, 16)] >>= do ss.
      >&<$_\text{\var{vend ss}}$ []
  \end{TASK}
  Sending $\Init$ to above task creates a new \var{vend} task
  which will ask for an amount of money and will result in a \var{Biscuit} or \var{Chocolate}.
  Note that this machine, contrary to the one in \cref{exm:vending-shared}, will run indefinitely.
\end{example}

\begin{figure}[h]
  \usemacro{G-Dynamic-Ext}
  \caption{Grammar for \DYNTOPHAT}
  \label{fig:dynamic-grammar}
\end{figure}


\subsection{Semantics}

Static and dynamic semantics of task pools are given in \cref{fig:semantics-dynamic},
and extended observations in \cref{fig:observations-dynamic}.

\begin{figure}[h]
  \begin{mathpar}
    \boxed{\RelationT} \\
    \userule{T-Pool} \\
    % \userule{E-Pool} \\
    \boxed{\RelationN} \\
    % \userule{N-Pool-Name} \\
    \userule{N-Pool} \\
    \boxed{\RelationH} \\
    \userule{H-Init} \\
    \userule{H-Kill} \\
    \userule{H-Pool} \\
  \end{mathpar}
  \caption{Semantic rules for \DYNTOPHAT}
  \label{fig:semantics-dynamic}
\end{figure}

\begin{figure}[h]
  \usemacro{O-Dynamic}
  \caption{Observations for \DYNTOPHAT}
  \label{fig:observations-dynamic}
\end{figure}


\subsection{Meta theory}

Propositions from \cite{Steenvoorden22} and \cite{Klijnsma2020} that still hold are the following:

\begin{itemize}
  \item Valued tasks are static
  \item Steady tasks stay steady
\end{itemize}

However, because of the dynamic nature of task pools, the following propositions need alteration:

\begin{itemize}
  \item Static tasks stay static
  \item Finished tasks stay finished
\end{itemize}

This is because task pools can always receive the initialise input ($\Init$),
so tasks containing a task pool will never get to a steady state!

We will discuss the impact on equational reasoning on tasks more thoroughly...