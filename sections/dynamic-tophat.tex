% !TEX root=../main.tex

\section{Dynamic TopHat}
\label{sec:dyntophat}

% \subsection{Language}

Although it is possible in \TOPHAT\ to start multiple tasks in parallel,
the amount of these tasks needs to be known in advance by the programmer.
Let us illustrate this with an example.

\begin{example}[Vending machine with shared stock]
  \label{exm:vending-shared}
  Take this slightly more complex vending machine,
  where two buyers can purchase a \typ{Snack} at the same time from a shared inventory.

  \begin{TASK}[emph={inventory,payment}]
    let vend' : Ref (List Snack) -> Task Snack = do inventory.
      enter Int >>? do payment.
      case payment of
        1 |-> decrease Biscuit inventory >>= do _.
             view Biscuit
        2 |-> decrease Chocolate inventory >>= do _.
             view Chocolate
        _ |-> fail
    let machine : Task (Snack * Snack) =
      share [(Biscuit, 24), (Chocolate, 16)] >>= do inventory.
      vend' inventory <&> vend' inventory
  \end{TASK}

  Here \fun{vend'} is similar to \fun{vend} from \cref{exm:vending-base},
  but now it takes a shared list with snacks as a parameter.
  \fun{decrease} is a function that decreases the amount of the given snack by one in the shared list of snacks.
  Task \fun{machine} first creates a shared \var{inventory},
  containing \con{Biscuit}s and \con{Chocolate}s,
  and starts two vending machines with these treats.
  This task results in a pair containing the two bought \typ{Snack}s.
\end{example}

We see that, at development time, we need to specify the number of \fun{vend} tasks that can be executed at the same time.
Although in the physical world, it is quite evident that one can only place a predefined amount of vending machines next to each other;
in the digital world, one would expect to initialize and kill such processes at any time.
To mitigate this shortcoming of \TOPHAT, we extend it with \emph{task pools}.


\subsection{Task pools}

\DYNTOPHAT\ is a moderate extension to the \TOPHAT\ base language where we add support for task pools.
Task pools contain zero or more instantiations of a \emph{template task} running in parallel.
In this regard, task pools are similar to the parallel-and combinator ($\Pair$).
End users can interact with all tasks currently running in a pool.
However, one can initiate a new instance of the template task by sending an \emph{init} event to the task pool.
Similarly, by sending a \emph{kill} event, the task pool terminates one of its children.

Language-wise, a task pool is a new combinator at the task level.
The combinator $\Pool^k_{t_0} []$ is parametrised by a unique key $k$, a template task $t_0$ and an (initially) empty task list $[]$.
Each time end-users send an init action to this pool: $k\Send\Init$, the template task is placed in the list of currently running tasks,
in this case resulting in $\Pool^k_{t_0} [t_0]$.
This way, an arbitrary amount of tasks can be started at runtime.
Tasks in the list can also be killed by sending a \emph{kill} action to the pool: $k\Send\Kill1$.
In this case, the first task in the pool will be killed, as indicated by the index 1.
The syntactic extensions to \TOPHAT\ are shown in \cref{fig:dynamic-grammar}.

\begin{figure}
  \usemacro{G-Dynamic-extensions}
  \caption{Grammar for \DYNTOPHAT\ extensions}
  \label{fig:dynamic-grammar}
\end{figure}

\begin{example}[Dynamic number of vending machines]
  \label{exm:vending-dynamic}
  Using the same \fun{vend'} task from \cref{exm:vending-shared},
  we can now create a vending machine where end-users can start an arbitrary amount of vending tasks
  by using \fun{vend'} as the template task of a pool.
  \begin{TASK}[emph={inventory}]
    let machine' : Task (List Snack) =
      share [(Biscuit, 24), (Chocolate, 16)] >>= do inventory.
      >&<$_{\fun{vend'}\,\var{inventory}}$ []
  \end{TASK}
  Note that now, \fun{machine'} returns a \typ{List} of \typ{Snack}s.
  Sending $\Init$ to \fun{machine'} creates a new \fun{vend'} task
  which will ask for an amount of money and will result in a \fun{Biscuit} or \fun{Chocolate}.
  One can create an arbitrary number of such tasks.

  Note that a standalone \fun{machine'}, contrary to the one in \cref{exm:vending-shared}, will run indefinitely
  in the same sense that an editor runs indefinitely.
  Adding a step to a continuation will break this cycle,
  as shown in the next example.
\end{example}

\begin{example}[Summarise sold snacks]
  \label{exm:summarise}
  We can write a task that starts our vending machine
  and only steps when the amount of sold snacks is exactly four.
  If that's the case, we show the amount of sold \con{Biscuit}s and \con{Chocolate}s.
  \begin{TASK}[emph={snacks}]
    let summarise =
      machine' >>= do snacks.
      if length snacks == 4 then
        let bs = show (count Biscuits snacks)
        let cs = show (count Chocolate snacks)
        view ("Sold " ++ bs ++ " biscuit(s) and " ++ cs " chocolate(s)")
      else fail
  \end{TASK}
\end{example}


\subsection{Semantics}

Typing rules of task pools are given in \cref{fig:typing-dynamic}.
Here we simply state that a task pool results in a list of values of type $\tau$
if every task in the pool, including the template task, is a $\Task$ of type $\tau$.

\begin{figure}
  \begin{mathpar}
    \boxed{\RelationT} \\
    \userule{T-Pool} \\
  \end{mathpar}
  \caption{Typing rules for \DYNTOPHAT}
  \label{fig:typing-dynamic}
\end{figure}

Extensions of the normalization relation of \TOPHAT\ is given in \cref{fig:semantics-dynamic-normalisation}.
Similar to the way normalization works for parallel-and (as shown in \cref{fig:semantics-normalisation}),
normalization of task pools normalizes each task in the pool and then constructs a new pool containing these normalized tasks.
Note that the template task is not normalized.
Otherwise, we would create unwanted sharing of editors.

\begin{figure}
  \begin{mathpar}
    % \userule{E-Pool} \\
    \boxed{\RelationN} \\
    \userule{N-Pool-Name} \\
    \userule{N-Pool} \\
  \end{mathpar}
  \caption{Normalisation rules for task pools}
  \label{fig:semantics-dynamic-normalisation}
\end{figure}

\begin{example}[Wrong! Unwanted sharing when normalizing template tasks]
  Take as an example the task pool $\Pool^{k_0}_{\Enter^\epsilon\Int} []$
  which would simply add editors to the pool asking to enter an integer.

  In case we would normalize the template task,
  we would have to attach a fresh key $k_1$ to the template task,
  resulting in:
  \begin{TASK}
    >&<$^{k_0}_{\Enter^{k_1}\Int}$ []
  \end{TASK}
  If we would now send $k_0\Send\Init$ twice,
  we would get:
  \begin{TASK}
    >&<$^{k_0}_{\Enter^{k_1}\Int}$ [enter$^{k_1}$ Int, enter$^{k_1}$ Int]
  \end{TASK}
  This is undesirable because it breaks our requirement that every editor should have a \emph{unique} key.
  It would be unclear what to do with the input $k_1\Send 42$.
  Therefore, we delay normalization of the template task until the moment they are added to the pool.
\end{example}

\begin{figure}[b]
  \begin{mathpar}
    \boxed{\RelationH} \\
    \userule{H-Init} \\
    \userule{H-Kill$_j$} \\
    \userule{H-Pool} \\
  \end{mathpar}
  \caption{Handling rules for task pools}
  \label{fig:semantics-dynamic-handling}
\end{figure}

Additional handling rules for \DYNTOPHAT\ are presented in \cref{fig:semantics-dynamic-handling}.
Rule \seerule{H-Init} shows that sending $k\Send\Init$ copies the template task to the pool.
Rule \seerule{H-Kill$_j$} removes the $j$th task from the pool on the input of $k\Send\Kill j$.
In both cases, the key $k$ in the input should be equal to the key of the task pool.
If that is not the case, rule \seerule{H-Pool} passes the input on to all tasks in the pool,
identical to \seerule{H-Pair} in \cref{fig:semantics-handling}.

Naturally, we also need to extend our failing, inputs, and value observation to deal with task pools.
\cref{fig:observations-dynamic} shows these additions.
The failing observation is $\False$ because
it is always safe to step to task pools.
After all, one can always initialize a new task in the pool.
The inputs observation collects inputs from every task in the pool
together with the $\Init$ input.
Additionally, it is possible to kill a task at every index in the pool.
As in previous work, it is easily proved that this is sound and complete with the handling rules presented in \cref{fig:semantics-dynamic-handling} \cite{Steenvoorden22}.
The value observation returns a list of values for every task in the pool.
If a task in the pool does not have a value (i.e., $\Value(n_j,\sigma) = \bot$ for some $j$),
this is simply not included in the list.
When there are no tasks in the pool, it results in the empty list.
Only in this way,
we can make a step from a task pool containing partial work.

\begin{figure}
  \usemacro{O-Dynamic}
  \caption{Observations for \DYNTOPHAT}
  \label{fig:observations-dynamic}
\end{figure}

\begin{example}[Partial work in pools]
  Take the \fun{summarise} task from \cref{exm:summarise}.
  Say we already sold 3 snacks, and two persons are trying to buy another snack.
  Our \fun{machine'} could be in the state
  \begin{TASK}
    >&<$_{\fun{vend'}\,\var{inventory}}$ [view Biscuit, view Chocolate,
      enter$^{k_3}$ Int >>? ..., view Chocolate, enter$^{k_5}$ Int >>? ...]
  \end{TASK}
  Note that, by definition of our value observation $\Value$,
  this task has the value \TS{[Biscuit, Chocolate, Chocolate]}.
  In particular, it does not result in $\bot$ because the third and fifth tasks do not have observable values.

  After user five pays the amount of 2 coins for a Chocolate (i.e. we send input $k_3\Send2$),
  above task will be rewritten to:
  \begin{TASK}
    >&<$_{\fun{vend'}\,\var{inventory}}$ [view Biscuit, view Chocolate,
      enter$^{k_3}$ Int >>? ..., view Chocolate, view Chocolate]
  \end{TASK}
  The value observation of this task would return the list \TS{[Biscuit, Chocolate, Chocolate, Chocolate]}.
  We now satisfy the guard in the definition of \fun{summarise} that the number of snacks should be four.
  We can take this step and immediately rewrite the whole task to summarise the different purchases that were made.
  The result will be:
  \begin{TASK}
    view ("Sold 1 biscuit(s) and 3 chocolate(s)")
  \end{TASK}
  This means the \fun{vend'} task of user three would be discarded.

  Note that we would not be able to make this step in the case that our value observation would result in $\bot$ when one of the values of the tasks in the pool would result in $\bot$.
\end{example}