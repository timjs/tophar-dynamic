% !TEX root=../main.tex

\section{Conclusion}
\label{sec:conclusion}

We presented the problem that programmers need to specify the number of parallel tasks in \TOPHAT\ during development time
and introduced \emph{task pools} as a solution.
In the resulting language \DYNTOPHAT\ is a moderate extension to \TOPHAT\ with more dynamic properties.
At runtime, end-users can initialise and kill tasks in a pool at will.
We presented the static and dynamic semantics of task pools extending \TOPHAT\ multiple semantic layers.
We were able to do this in such a way that the impact on formal reasoning of \TOPHAT\ programs is minimal.
Although symbolic execution could end in an infinite loop,
we altered our next-step hint generation system so that it can still support end-users during workflow execution.
Also, our semantic extensions were defined in such a way, that equational reasoning on tasks is still possible
and transformation laws proved in earlier work still hold.

This reinforces our belief that the given definitions and semantics for task pools are the right choice.
