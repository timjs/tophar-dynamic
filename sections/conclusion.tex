% !TEX root=../main.tex

\section{Conclusion}
\label{sec:conclusion}

We started with the problem that programmers need to specify the number of parallel tasks in \TOPHAT\ during development time.
As a solution, we extended the language with \emph{task pools}.
The resulting language \DYNTOPHAT\ is a moderate extension to \TOPHAT\ with more dynamic properties.

At runtime, end-users can initialize and kill tasks in a pool at will.
We presented the static and dynamic semantics of task pools by extending \TOPHAT's multiple semantic layers
and compared this to an alternative approach.
We were able to extend the language in such a way that the impact on formal properties of \TOPHAT\ programs is minimal.

Although symbolic execution could end in an infinite loop,
we altered our next-step hint generation system so that it can still support end-users during workflow execution.
Also, our semantic extensions were defined in such a way, that equational reasoning on tasks is still possible
and transformation laws proved in earlier work still hold.
This reinforces our belief that the presented definitions and semantics for task pools are the right choice.
