% !TEX root=../main.tex

\subsection{Symbolic execution}
\label{sec:symbolic}

Previous work introduced a symbolic execution semantics for \TOPHAT\ programs~\cite{conf/ifl/NausSK19,Naus20,Steenvoorden22}.
By defining semantic rules that introduce symbols where user input is expected, the behaviour of a \TOP\ program can be fully simulated.
Simulation of tasks keeps track of \emph{path conditions}.
By sending these path conditions to \SMT\ solvers,
simulations can be used to reason over the properties of the program.

Every concrete semantic relation from \cref{fig:semantic-layers} has a symbolic counterpart.
Notation for these relations is summarised in \cref{tab:symbolic-relations}.
Compared to the concrete relations,
tasks are augmented with a path condition $\phi$.
For labelled relations, instead of taking inputs $\iota$ written above the arrow,
simulation rules generate inputs $\iota'$, which are written below the arrow.
\cref{fig:symbolic-semantics} lists a selected number of symbolic execution rules from \TOPHAT.

\begin{table}
  \begin{tabular}{lRCLRCL}
    \toprule
                & \multicolumn{3}{c}{Concrete}                            & \multicolumn{3}{c}{Symbolic} \\
    \midrule
    evaluate    & e               & \evaluate        & v                  & e                       & \symevaluate        &  v|_{\phi'}                 \\
    normalize   & t,\sigma        & \normalize       & n',\sigma',\delta' & t|_{\phi},\sigma        & \symnormalise       & n'|_{\phi'},\sigma',\delta' \\
    fixate      & t,\sigma,\delta & \fixate          & n',\sigma'         & t|_{\phi},\sigma,\delta & \symfixate          & n'|_{\phi'},\sigma'         \\
    handle      & n,\sigma        & \handle{\iota}   & t',\sigma',\delta' & n|_{\phi},\sigma        & \symhandle{\iota}   & t'|_{\phi'},\sigma',\delta' \\
    interact    & n,\sigma        & \interact{\iota} & n',\sigma'         & n|_{\phi},\sigma        & \syminteract{\iota} & n'|_{\phi'},\sigma'         \\
    \bottomrule
\end{tabular}

  \caption{Semantic relations for concrete and symbolic execution}
  \label{tab:symbolic-relations}
\end{table}

As with end-user input,
symbols enter the simulation at editors.
As an example, rule \seerule{SH-Enter} generates a fresh symbol $z'$, returns a valued editor $\Update^k z'$
and leaves the path condition $\phi$ unaltered.
For a pair of tasks, instead of one rule, the symbolic semantics contains two rules:
one for generating input for the left task (\seerule{SH-Pair-Left}), and one for the right (\seerule{SH-Pair-Right}).
This makes the symbolic semantics non-deterministic.

\begin{figure}
  \begin{mathpar}
    \boxed{\RelationSH} \\
    \usemacro{SH-Enter} \\
    \usemacro{SH-Pair-Left}
    \usemacro{SH-Pair-Right} \\
  \end{mathpar}
  \caption{Selected symbolic handling rules}
  \label{fig:symbolic-semantics}
\end{figure}

For \DYNTOPHAT\ we extend symbolic handling relation with rules for task pools.
As task pools are in essence a combination of editors and parallel-or,
the symbolic rules are alike to the ones shown in \cref{fig:symbolic-semantics}.
First, for every possible subtask in a task pool,
we add rule \refrule{SH-Pool$_j$} to simulate the $j$-th subtask in a pool.
This is comparable to rules \refrule{SH-Pair-\{Left,Right\}}.
Then, also for every possible subtasks,
we add rule \refrule{SH-Pool-Kill$_j$} to kill the $j$-th subtask.
Last, we add a rule to symbolically initialise a new task in the pool: \refrule{SH-Pool-Init}.

\begin{figure}
  \begin{mathpar}
    \boxed{\RelationSH} \\
    \usemacro{SH-Pool-Init} \\
    \usemacro{SH-Pool-Kill$_j$} \\
    \usemacro{SH-Pool$_j$} \\
  \end{mathpar}
  \caption{Symbolic rules for \DYNTOPHAT}
  \label{fig:symbolic-semantics-dynamic}
\end{figure}

Top-level symbolic interaction of tasks is written $\RelationSI$,
meaning that a normalised task $n$ under path condition $\phi$ in state $\sigma$ generates input $\iota$, a new normalised task $n'$ under path condition $\phi'$ in state $\sigma'$.
Rule \seerule{SI-Handle} initiates simulation
by using the symbolic handling and fixation semantics.
Fixation, as in the concrete case, resolves any dependencies between shared heap locations.

\begin{figure}
  \begin{mathpar}
    \boxed{\RelationSI} \\
    \usemacro{SI-Handle} \\
  \end{mathpar}
  \caption{Symbolic interaction rule}
  \label{fig:symbolic-semantics}
\end{figure}

\todo{alinea herschrijven zodat het bij voorgaande past}
Ultimately, the symbolic execution of a \TOPHAT\ program is then powered by a top-level simulation function, that leverages the symbolic rules to perform symbolic execution.
Since a user can update an editor as often as they want, special care is taken to ensure that only \textit{productive} updates are generated by the symbolic execution function.
An editor can only be updated twice, if the task, modulo the updated value, has changed.
This allows exactly enough repetition to capture the full program behaviour.
For example, the second task in a parallel composition might rely on a value being updated in the first task, to make progress.

When it comes to \DYNTOPHAT, there is an additional opportunity for infinite interaction, namely to spool up a new task.
Unfortunately, in this case, there is no way to set an upper bound on this interaction without changing the program's behaviour.
\todo{voorbeeld concreet maken in syntax?}
Imagine we have a task pool combinator, followed by a step.
We restrict the number of new task initializations to be no larger than some $n$.
The step could very well require $n+1$ values before it will allow a step.
It is clear from this example that limiting task initialization is not a suitable way to deal with the infinite interaction coming from the pool combinator.

The pool combinator breaks traditional symbolic execution as introduced in previous work~\cite{conf/ifl/NausSK19}.
That does not mean that the symbolic execution of \DYNTOPHAT\ programs is now completely useless.
Built on top of the symbolic execution engine, \citet{conf/sfp/NausS20} presents a system that assists users of \TOPHAT\ programs in finding the next input action to perform.
The basic idea is that symbolic execution can be leveraged to reach some goal $\psi$.
\cref{fig:next-step-hints} defines a function that calculates these so-called next-step-hints.

\begin{figure}
  \begin{mathpar}
    \usemacro{O-Hints}
  \end{mathpar}
  \caption{Next-step hint generation function}
  \label{fig:next-step-hints}
\end{figure}

\todo{explain the function in this figure, and how it uses symbolic execution.}
The addition of a symbolic execution rule for the task pool combinator breaks the definition from \cref{fig:next-step-hints}, since the symbolic interaction \textit{traces} are now potentially infinite.
Instead of calculating all possible hints, one could alter the hints function to perform a breadth-first search through the now infinite trace space.
This allows us to return a hint that guides the user to the shortest path to their goal.
\cref{fig:next-step-hints-new} lists an updated definition of the next-step hint function.



\begin{figure*}
  \begin{mathpar}
    \usemacro{O-Generate}
  \end{mathpar}
  \caption{Altered next-step hint generation function}
  \label{fig:next-step-hints-new}
\end{figure*}

Coming back to reasoning about task properties, the breadth-first search strategy can also be applied to finding assertion violations.
\citet{DBLP:conf/tap/NausVSR23} demonstrate a breadth-first search approach to finding assertion violations in low-level programs with potential infinite search spaces.
This approach can also be applied to \DYNTOPHAT\ programs, using the insights described above.

To conclude, adding task pools to \TOPHAT\ does restrict the use of symbolic execution, but it still has several viable applications.
These restrictions are similar to those found with any real-world application of symbolic execution.
A more extensive discussion on this point can be found elsewhere~\cite{conf/ifl/NausSK19}.