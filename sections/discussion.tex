% !TEX root=../main.tex

\section{Discussion}
\label{sec:discussion}

One could think of multiple alternative implementations of task pools
by choosing different semantics or by using already existing \TOPHAT\ language constructs.

\paragraph{Alternative semantics}

\cref{sub:dynamic-semantics} showed our decision for the value observation on task pools to ignore unvalued children.
Another option would have been to make task pools unvalued when they contain an unvalued child.
However, we would not be able to specify partial work, as exemplified in \cref{exm:partial-work}.
Also, our equivalence properties would change.
\cref{fig:task-conditions} would require an additional arrow from the \emph{unsteady} to the \emph{running} condition,
as task pools could lose their observed value after adding a new (unvalued) task to the pool.
Property 2b in \cref{sub:task-properties} would not naturally hold anymore.

In the semantics of \seerule{H-Init} we could have chosen to send values along the init action ($k \Send \Init b$) to parametrize task initialization.
This could be accomplished by requiring a template \emph{function} $f_0$ instead of a template \emph{task} $t_0$ for each pool.
Rule \seerule{H-Init} could be extended to dynamically check if the given actual parameters fit the types of the required formal ones,
similar to the type check of \seerule{H-Enter}.
However, a similar interaction could be programmed by asking end-users for this information in the first step of template task $t_0$
without a lot of extra semantics.

Initializing task pools with a fixed size would prohibit symbolic execution from non-termination,
as there would be a maximum on the number of times the symbolic engine could initialize a new task.
However, it would not prohibit the symbolic engine from generating infinite input sequences
where one or more initializations are followed by killing the same or previous tasks.

\paragraph{Utilizing existing TopHat features}

Task pools could be implemented with a shared reference to a list of running tasks,
and on top of that, creating custom tasks to remove children by their index
or append the template task to this list.
Packing these operations together using an infinite loop would seemingly give us a working implementation:
\begin{TASK}
  let kill = do ts. enter Int >>= do i. remove i ts
  let init = do t. do ts. enter Unit >>= do _. append t ts
  let pool = do t.
    share [] >>= do ts.
    forever (watch h <&> kill ts <&> init t ts)
\end{TASK}
This is effectively the way that \ITASKS\ implements dynamic task lists \cite{conf/pepm/PlasmeijerAKLNG11}.
However, there are two reasons this is not a good fit for \TOPHAT.

The first is that the forever combinator ($\Forever$), is not definable in \TOPHAT.
Our goal is to keep our host language total.
The combinator could be added explicitly to the task layer,
but this would impose severe restrictions on the symbolic execution engine,
in the same way, non-total functions would do.

The second is that it would require references and editors over the \typ{Task} type constructor.
Currently, both are restricted to basic types $\beta$,
because basic types are trivially displayable, parsable and equatable.
Because $\beta$ does not include references, tasks, and, most importantly, functions,
the inductive base cases in the equivalence theory of \citet{conf/sfp/KlijnsmaS22} can suffice with trivial data equality.

This raises the question if tasks are more like basic types, or if they should behave more like functions.
What would $\Update (\Update 42)$ mean?
How can end-users dynamically enter a task that results in an integer?
Could they create new tasks at will?
Or should they pick from a list?
Is this list predefined or is it populated dynamically during runtime?
Although \ITASKS\ supports editors on tasks,
the above example cannot be rendered at runtime:
\ITASKS\ renders is the string "(empty)" in a dashed box.
The equality on two tasks in \ITASKS\ is always \var{True}.

Therefore, we think explicitly adding task pools to \TOPHAT\ is a better fit for a language whose main goal is formal reasoning about task-oriented programs.
