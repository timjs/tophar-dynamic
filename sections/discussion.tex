% !TEX root=../main.tex

\section{Discussion}

% \subsection{Alternative implementation}

One could think of an alternative implementation of task pools,
using already existing \TOPHAT\ language constructs.
A possible solution could be by using a shared reference to a list of running tasks,
and creating custom tasks to remove a task by its index from this list
or append the template task to this list.
\begin{TASK}
  let kill = do ts. enter Int >>= do i. remove i ts
  let init = do t. do ts. enter Unit >>= do _. append t ts
  let pool = do t.
    share [] >>= do ts.
    forever (watch h <&> kill ts <&> init t ts)
\end{TASK}

This is effectively the way that \ITASKS\ implements dynamic task lists \cite{conf/pepm/PlasmeijerAKLNG11}.
However, there are two reasons this is not a good fit for \TOPHAT.

The first is that the forever combinator ($\Forever$), is not definable in \TOPHAT.
Our goal is to keep our host language total.
The combinator could be added explicitly to the task layer,
but this would impose severe restrictions on the symbolic execution engine,
in the same way non-total functions would do.

The second is that it would require references and editors over the $\Task$ type constructor.
Currently, both are restricted to basic types,
because basic types are trivially displayable, parsable and equatable.
This last property is used by \cite{Klijnsma2020} as a basis to reason about task equivalence.

This raises the question if tasks are more like basic types, or if they should behave more like functions.
What would $\Update (Update\ 42)$ mean?
How can end users dynamically enter a task that results in an integer?
Could they create new tasks at will?
Or should they pick from a list?
Is this list predefined or is it populated dynamically during runtime?
Although \ITASKS\ supports editors on tasks,
above example cannot be rendered at runtime and the equality on two tasks in \ITASKS\ is always \var{True}.

Therefore, we think explicitly adding task pools to \TOPHAT\ is a better fit for a language which main goal is formal reasoning about task-oriented programs.
