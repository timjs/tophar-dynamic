% !TEX root=../main.tex

\section{Introduction}
\label{sec:introduction}

In modern society, it is almost unthinkable to run any organization without the help of software that manages the organization's processes.
From managing customers and patients to monitoring systems and situations, software has become a crucial component.

Traditionally, these systems are developed using tools that offer a rather low level of abstraction.
Standard Object-Oriented approaches allow some superficial domain modelling and best practices like the Unified Process from the late 90's provide developers with some guidance.

Many researchers found that more powerful abstractions are needed to develop these all-too-common systems in a way that results in more robust and reliable systems.
Since then, a lot of work has been published on better describing and implementing business processes, such as the workflow patterns by \citeauthor{journals/dpd/AalstHKB03}, Workflow Nets \cite{journals/jcsc/Aalst98}, and business process calculi like \BPEL\ \cite{bpel}.

Most notable is the work by \citet{conf/ifl/KoopmanPA08} and \citet{conf/ppdp/PlasmeijerLMAK12} on \ITASKS.
They present an abstraction over workflow systems implemented as a \DSL\ in the functional programming language \CLEAN\ \cite{plasmeijer2002clean}.
The idea behind their paradigm is to model collaboration patterns, and to abstract away from things like \GUI, client-server communication, and databases.
Their work proved to be successful, with several extensions and the technology now being used in a spin-off company \cite{com/tss/viia}.

Even though \ITASKS\ is implemented as a \DSL\ in a functional programming language, the paradigm lacks any formal semantics.
This heavily restricts the amount of formal reasoning that can be done on \ITASKS\ programs.

To overcome this downside, \TOPHAT\ was introduced~\cite{conf/ppdp/SteenvoordenNK19}.
Built on the simply typed lambda calculus,
\TOPHAT\ is a \TOP\ implementation that has a complete formal semantics.
Several works leverage this to perform symbolic execution \cite{conf/ifl/NausSK19},
generate next-step hints for end-users \cite{conf/sfp/NausS20},
and reason about equivalence of tasks \cite{conf/sfp/KlijnsmaS22}.
% A complete and thorough description of all of \TOPHAT's features can be found in \citet{Steenvoorden22}.

Compared to the \ITASKS\ system, \TOPHAT\ is more restrictive in managing tasks itself.
\TOPHAT\ programs can only define a static amount of work, whereas \ITASKS\ programs can dynamically spool up new tasks and terminate old ones.
In this paper, we introduce \DYNTOPHAT, an extension to \TOPHAT\ that allows end-users to initialize and kill tasks at runtime.
Key in our approach is that we do not compromise the formal properties when adding this feature.

% \paragraph{Contributions}

Our main contributions are as follows.
\begin{itemize}
  \item
    We extend the basic \TOPHAT\ language with dynamic \emph{task pools},
    allowing end-users to initialize and kill tasks at runtime.
    We call this extension \DYNTOPHAT.
  \item
    We reason that symbolic execution of tasks containing pools can result in infinite simulation.
    However, we show that it is still possible to use symbolic execution to generate next-step hints for end-users
    by altering our algorithm from previous work.
  \item
    We prove that the addition of task pools has a small impact on the metatheoretical properties of \TOPHAT.
    We can still reason about the evolution of tasks at runtime
    and prove when two tasks are equal.
    Also, previously published transformation laws on tasks still hold:
    the \typ{Task} operation on types is a functor but cannot be a monad.
\end{itemize}

% \todo{Say something about validation and proofs here?}

% \paragraph{Structure}

The remainder of this paper is structured as follows.
\cref{sec:tophat} introduces the \TOPHAT\ language as presented in \citet{Steenvoorden22}.
\cref{sec:dyntophat} then goes on to present our dynamic version of \TOPHAT.
% After having introduced all language constructs, we study a bigger example in \cref{sec:example}.
We then study the implications on \TOPHAT's properties in \cref{sec:properties}.
We discuss the impact of task pools on the symbolic execution mechanism and task equivalence.
Related work is discussed in \cref{sec:relatedwork}.
After a short discussion on the design decisions of \DYNTOPHAT\ in \cref{sec:discussion},
we conclude in \cref{sec:conclusion}.
