% !TEX root=../main.tex

\section{Introduction}

In modern society, it is almost unthinkable to run any organization without the help of software that manages the oganization's processes.
From managing customers and pations to monitoring systems and situations, software has become a crucial component.

Traditionally, these systems are developed using tools that offer a rather low level of abstraction.
Standard Object Oriented appraches allow some superficial domain modeling, and best practices like the Unified Process from the late 90's provide developers with some guidance.

However, many researchers found that more powerful abstractions are needed to develop these all too common systems in a way that results in more robust and reliable systems.
Since then, a lot of work has been published on better describing and implementing business processes, such as the workflow patterns by van der Aalst et al., Worflow nets, (INSERT SOME MORE HERE).
Most notably is the by Koopman et al. and Plasmeijer et al. on iTasks.
They present an abstraction over workflow systems implemented as a DSL in the functional programming language CLEAN.
The idea behind their paradigm is to model collaboration patterns, and to abtract away from things like GUI, client-server communication, and databases, 
Their work proved to be succesful, with several extensions and the technology now being used in a spin-off company.

Even though iTasks is implemented as a DSL in a functional programming language, the paradigm lacks any formal semantics.
This heavily restricts the amount of formal reasoning that can be done on iTasks programs.

To overcome this downside, TopHat was introduced.
Built on top of the lambda-calculus, TopHat is a TOP implementation that has a completely formal semantics.
Several works leverage this to perform symbolic execution, and (SAY MORE).
