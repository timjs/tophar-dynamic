% !TEX root=../main.tex

\section{Introduction}
\label{sec:introduction}

In modern society, it is almost unthinkable to run any organization without the help of software that manages the organization's processes.
From managing customers and patients to monitoring systems and situations, software has become a crucial component.

Traditionally, these systems are developed using tools that offer a rather low level of abstraction.
Standard Object-Oriented approaches allow some superficial domain modeling and best practices like the Unified Process from the late 90's provide developers with some guidance.

However, many researchers found that more powerful abstractions are needed to develop these all-too-common systems in a way that results in more robust and reliable systems.
Since then, a lot of work has been published on better describing and implementing business processes, such as the workflow patterns by van der Aalst et al., Workflow nets, (INSERT SOME MORE HERE).
Most notably is the by Koopman et al. and Plasmeijer et al. on iTasks.
They present an abstraction over workflow systems implemented as a DSL in the functional programming language CLEAN.
The idea behind their paradigm is to model collaboration patterns, and to abstract away from things like GUI, client-server communication, and databases,
Their work proved to be successful, with several extensions and the technology now being used in a spin-off company.

Even though iTasks is implemented as a DSL in a functional programming language, the paradigm lacks any formal semantics.
This heavily restricts the amount of formal reasoning that can be done on iTasks programs.

To overcome this downside, TopHat was introduced.
Built on top of the lambda calculus, TopHat is a TOP implementation that has a completely formal semantics.
Several works leverage this to perform symbolic execution, and (SAY MORE).

Compared to the original iTasks system, TopHat is more restrictive in managing tasks itself.
TopHat programs can only define a static amount of work, whereas iTasks programs can dynamically spool up new tasks and terminate old ones.
In this paper, we introduce \DYNTOPHAT, an extension to \TOPHAT\ which allows end users to initialize and kill tasks at runtime.
Key in our approach is that we do not compromise any of the formal properties when adding this feature.

(Maybe some more text on what our system can and cannot do).

(SAY SOMETHING ABOUT VALIDATION HERE???)

Section~\ref{sec:tophat} introduces the \TOPHAT\ language as presented in earlier work. Section~\ref{sec:dyntophat} then goes on to present our dynamic version of \TOPHAT.
Related work is discussed in~\ref{sec:relatedwork}. After a short discussion on the implications of \DYNTOPHAT in Section~\ref{sec:discussion}, we conclude in Section~\ref{sec:conclusion}.

\begin{TASK}[emph={passengers,seats,free}]
  let main = do
    free <- share [(r, s) | r <- [1 .. 10], s <- ['A' .. 'D']]
    watch free <& pool (book free)

  let book free = do
    enter List Passenger >>= do passengers.
      all valid passengers |->
        choose_seats passengers free >>= seats.
        confirm_booking passengers seats >>= _.
        assert length passengers == length seats
        assert unique seats
        assert seats /< free

  let choose_seats = do passengers free.
    watch free <&> enter Seats >>= do (fs, seats).
    seats fs /\ length passengers == length seats
      |-> confirm_booking passengers seats free

  confirmBooking :: Passengers -> Seats -> Store h Seats -> Task h (Passengers, Seats)
  let confirm_booking == do passengers seats free
    free <<= difference seats
    view (passengers, seats)
\end{TASK}

  % allValid :: Passengers -> Bool
  % allValid ps = all isValid ps && any isAdult ps

  % isValid :: Passenger -> Bool
  % isValid (n, a) = n /= "" && a >= 0

  % isAdult :: Passenger -> Bool
  % isAdult (_, a) = a >= 18

  % isCorrect :: Seats -> Seats -> Bool
  % isCorrect ss fs = do
  %   List.intersect ss fs == ss
