% !TEX root=../main.tex

\section{Introduction}
\label{sec:introduction}

In modern society, it is almost unthinkable to run any organization without the help of software that manages the organization's processes.
From managing customers and patients to monitoring systems and situations, software has become a crucial component.

Traditionally, these systems are developed using tools that offer a rather low level of abstraction.
Standard Object-Oriented approaches allow some superficial domain modelling and best practices like the Unified Process from the late 90's provide developers with some guidance.

However, many researchers found that more powerful abstractions are needed to develop these all-too-common systems in a way that results in more robust and reliable systems.
Since then, a lot of work has been published on better describing and implementing business processes, such as the workflow patterns by \citeauthor{journals/dpd/AalstHKB03}, Workflow Nets \cite{journals/jcsc/Aalst98}, and business process calculi like \BPEL\ \cite{bpel}.

Most notably is the work by \citet{conf/ifl/KoopmanPA08} and \citet{conf/ppdp/PlasmeijerLMAK12} on \ITASKS.
They present an abstraction over workflow systems implemented as a \DSL\ in the functional programming language \CLEAN\ \cite{plasmeijer2002clean}.
The idea behind their paradigm is to model collaboration patterns, and to abstract away from things like \GUI, client-server communication, and databases.
Their work proved to be successful, with several extensions and the technology now being used in a spin-off company \cite{com/tss/viia}.

Even though \ITASKS\ is implemented as a \DSL\ in a functional programming language, the paradigm lacks any formal semantics.
This heavily restricts the amount of formal reasoning that can be done on \ITASKS\ programs.

To overcome this downside, \TOPHAT\ was introduced.
Built on top of the simply typed lambda calculus,
\TOPHAT\ is a \TOP\ implementation that has a completely formal semantics.
Several works leverage this to perform symbolic execution \cite{conf/ifl/NausSK19},
generate next-step hints for end users \cite{conf/sfp/NausS20},
and reason about equivalence of tasks \cite{conf/sfp/KlijnsmaS22}.
% A complete and thorough description of all of \TOPHAT's features can be found in \citet{Steenvoorden22}.

Compared to the original \ITASKS\ system, \TOPHAT\ is more restrictive in managing tasks itself.
\TOPHAT\ programs can only define a static amount of work, whereas \ITASKS\ programs can dynamically spool up new tasks and terminate old ones.
In this paper, we introduce \DYNTOPHAT, an extension to \TOPHAT\ which allows end users to initialize and kill tasks at runtime.
Key in our approach is that we do not compromise any of the formal properties when adding this feature.
% \subsection{Contributions}

\todo{Maybe some more text on what our system can and cannot do}
\todo{Say something about validation here}

\paragraph{Structure}

\cref{sec:tophat} introduces the \TOPHAT\ language as presented in \citet{Steenvoorden22}.
\cref{sec:dyntophat} then goes on to present our dynamic version of \TOPHAT.
% After having introduced all language constructs, we study a bigger example in \cref{sec:example}.
Then we study the implications on \TOPHAT's properties in \cref{sec:properties}.
We discuss the impact of dynamic tasks on the symbolic execution mechanism, task equivalence, and visualisation of tasks.
Related work is discussed in \cref{sec:relatedwork}.
After a short discussion on the implications of \DYNTOPHAT\ in \cref{sec:discussion},
we conclude in \cref{sec:conclusion}.
