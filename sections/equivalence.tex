\newmacro{citeprop}[1]{[Prop.~#1]} % {\cite[Prop.~#1]{conf/sfp/KlijnsmaS22}}

\subsection{Task equivalence}
\label{sec:equivalence}

Reasoning about task equivalence is useful for two main purposes.
First, we can argue that two (parts of) workflows, as specified by developers,
actually have the same impact and are interchangeable in the code.
Second, we can prove several general transformation laws for tasks,
which aid in the refactoring of task-oriented code.
%
In previous work, we investigated the equivalence of tasks
and we proved some transformation laws.
We concluded that the $\Task$ operation on types is a functor,
but cannot be a monad \cite{conf/sfp/KlijnsmaS22}.

To achieve this,
we categorise tasks based on the observations we can make on them.
The observations we use, the ones discussed in \cref{sec:observations}:
are if the task is failing ($\Failing$),
if the task can receive any inputs ($\Inputs$),
and if the task has a value ($\Value$).
Based on these three observations,
a (fully fixated) task can be in one of five \emph{conditions} which we will summarise below.
We refer to the propositions as proved in \citet{conf/sfp/KlijnsmaS22}.

\begin{description}

  \item[Failing]
    A failing task $t$ is a task for which the failing function $\Failing(t)$ yields $\True$.
    If a task is failing, we can prove that it accepts no more user input and has no value \citeprop{12}.
    That is, $\Inputs(t) = \nothing$ and $\Value(t,\sigma) = \bot$.
    We also know that, if $\Failing(t)$ holds, it stays failing,
    simply because there is no input that can rewrite $t$ to another $t'$ \citeprop{1}.

  \item[Stuck]
    A stuck task $t$ is a task which does not fail, does not accept user input, and does not have a value.
    That is, $\Failing(t) = \False$, $\Inputs(t) = \nothing$, and $\Value(t,\sigma) = \bot$.
    Similar to failing tasks, stuck tasks will always remain stuck because no more user interaction is possible and, by assumption, it is already fully fixated \citeprop{3}.

  \item[Steady]
    If a task $t$ has a value but it cannot handle any input, we call it \emph{steady}.
    An example of a steady task is $\View^k 42$.
    Because this task accepts no more user input, its value will always remain equal to $42$.
    So steady tasks will stay steady \citeprop{2a}.

  \item[Unsteady]
    If a task $t$ has a value and it can handle input, we call it \emph{unsteady}.
    An example of an unsteady task is $\Update^k 42$.
    This task still accepts user input, and thus its value can keep on changing.
    Even though $\Update^k 42$ and $\View^k 42$ yield the same value,
    they should not be equivalent, since one's value can be changed and the other's cannot.
    If a task $t$ is unsteady, it will never terminate: an unsteady task will always remain unsteady with the same input space, only its value may change \citeprop{2b}.
    We do say, however, that both steady and unsteady tasks are \emph{finished}.

  \item[Running]
    The last condition is when a task $t$ does not fail, does not have a value, but still accepts user input.
    Because there is still user interaction possible, it may be the case that, with the right input,
    it transitions to one of the previously described task conditions.
    The simplest example of a running task is the empty editor $\Enter^k \beta$, which becomes an unsteady task once it receives a valid input event.
    Although running tasks can keep running forever,
    we at least can prove that they will not fail \citeprop{4}.

\end{description}

These five conditions are proved to be mutually exclusive \citeprop{5}.
\cref{tab:task-conditions} summarises all conditions
and shows some more example tasks in each condition.
The possible transitions for tasks from one condition to another are drawn in \cref{fig:task-conditions}.

\begin{table}
  \begin{tabular}{CCClL}
  \toprule
  \Failing   & \Inputs    & \Value     & \alert{Condition} & \alert{Examples}                                                                                                             \\
  \midrule
  \checkmark & -          & -          & Failing           & \Fail, \,\, \Fail \Pair \Fail, \,\, (\lambda x.x) \Trans \Fail, \,\, \Fail \Step \lambda x.\View x \\
  -          & -          & -          & Stuck             & \View 2 \Step \lambda x.\Fail, \,\, \View 2 \Pair \Fail, \,\, \Fail \Pair \View 2                                            \\
  -          & -          & \checkmark & Steady            & \View 2, \,\, \View 2 \Pair \View 3, \,\, \View \tuple{2, 3}                          \\
  -          & \checkmark & \checkmark & Unsteady          & \Update 2, \,\, \Update 2 \Pair \Update 3, \,\, \Update \tuple{2, 3}                                                         \\
  -          & \checkmark & -          & Running           & \Enter \Int \Pair \Fail, \,\, \Update 2 \Step \lambda x.\Fail                                                                \\
  \bottomrule
\end{tabular}
\\
{\small
  Checkmarks for $\Failing$, $\Value$ and $\Inputs$ for a task $t$ and a state $\sigma$ indicate that\\
  $\Failing(t) = \True$,
  $\Value(t, \sigma) = v$ for $v \neq \bot$, and
  $\Inputs(t) \neq \nothing$ respectively.
}

  \caption{Conditions for tasks}
  \label{tab:task-conditions}
\end{table}

\begin{figure}
  \begin{tikzpicture}[x=1pc,y=1pc,shorten >=1pt,minimum size=3.5pc,node distance={0.33pc and 3pc},>={Triangle},draw=black,thick]
  \node[circle,draw,very thick] (running)                            {\emph{Running}};
  \node[circle,draw,very thick] (failing)   [above left=of running]  {\emph{Failing}};
  \node[circle,draw,very thick] (stuck)     [above right=of running] {\emph{Stuck}};
  \node[circle,draw,very thick] (finished)  [below left=of running]  {\emph{Steady}};
  \node[circle,draw,very thick] (recurring) [below right=of running] {\emph{Unsteady}};
  %
  % \draw[thick,->] (running) -- (failing);
  \draw[->] (running) -- (finished);
  \draw[->] (running) -- (recurring);
  \draw[->] (running) -- (stuck);
  \draw[->] (recurring) edge [out=423,in=387,looseness=6]  (recurring);
  \draw[->] (running) edge [out=108,in=72,looseness=6]  (running);
\end{tikzpicture}

  \caption{Possible task conditions and their transitions, where a transition is caused by user interaction ($\interact{\iota}$) for some input $\iota$.}
  \label{fig:task-conditions}
\end{figure}

Next, we discuss how this theory changes after the addition of task pools.
It is always possible to send initialise a new task in a task pool.
As shown in \cref{fig:observations-dynamic},
the inputs observation $\Inputs$ on a task pool $\Pool_{t_0}^k [\many{n}]$ will contain the input $k\Send\Init$
regardless of the tasks in the pool.
This has a consequence for the diagram in \cref{fig:task-conditions}.

\begin{example}{Integer task pool}
  Take for example the simple task $t = \Pool_{\Enter^\epsilon \Int}^k_0 [\Update^k_1 42]$.

  By definition of $\Value$, it has an observable value, namely the singleton list $[42]$.
  By definition of $\Inputs$, it also has possible inputs,
  which is the set $\set{k_0\Send\Init, k_0\Send\Kill1, k_1\Send x \Mid x\in\Int}$.
  This means we can change the value in editor keyed $k_1$,
  which will also change the value of the whole task pool,
  but we can also kill or initialise tasks in the pool.

  Say, we send $k_0\Send\Init$ to $t$.
  Then, $t$ will be rewritten to $\Pool_{\Enter^\epsilon \Int}^k_0 [\Update^k_1 42, \Enter^k_2\Int]$.
  Because $\Enter^k_2\Int$ does not have a value,
  the whole pool does not have a value any more!
  \todo{is this the semantics we want?}
  This means $t$ moved from condition unsteady to running.
\end{example}

Propositions stating that failing tasks stay failing \citeprop{1},
steady tasks stay steady \citeprop{2a},
and stuck tasks stay stuck \citeprop{3},
stay the same.
Also, by construction, running tasks will not fail \citeprop{4}.
What will change, however, is Prop.~2b:
unsteady tasks will not stay unsteady with the same input space after incorporating task pools.

---------






Propositions from \cite{Steenvoorden22} and \cite{Klijnsma2020} that still hold are the following:

\begin{itemize}
  \item Valued tasks are static
  \item Steady tasks stay steady
\end{itemize}

However, because of the dynamic nature of task pools, the following propositions need alteration:

\begin{itemize}
  \item Static tasks stay static
  \item Finished tasks stay finished
\end{itemize}

This is because task pools can always receive the initialise input ($\Init$),
so tasks containing a task pool will never get to a steady state!

We will discuss the impact on equational reasoning on tasks more thoroughly...