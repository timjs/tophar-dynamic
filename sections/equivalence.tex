\newcommand{\citeprop}[1]{} % {[Prop.~#1]} % {\cite[Prop.~#1]{conf/sfp/KlijnsmaS22}}

\subsection{Task equivalence}
\label{sec:equivalence}

Reasoning about task equivalence is useful for two purposes.
First, we can argue that two (parts of) workflows
actually have the same impact and are interchangeable in the code.
Second, we can prove several general transformation laws for tasks,
which aid in the refactoring of task-oriented code.
%
In previous work, we investigated the equivalence of tasks
and we proved some transformation laws.
We concluded that the \typ{Task} operation on types is a functor,
but cannot be a monad \cite{conf/sfp/KlijnsmaS22}.

To be able to make statements about task equivalence,
we categorize tasks based on the observations we can make on them.
The observations we use are the ones discussed in \cref{sec:observations}:
if the task is failing ($\Failing$),
if the task can receive any inputs ($\Inputs$),
and if the task has a value ($\Value$).
Based on these three observations,
a (fully fixated) task can be in one of five \emph{conditions} which we will summarise below.
% We refer to propositions as proved in \citet{conf/sfp/KlijnsmaS22}.

\begin{description}

  \item[Failing]
    A failing task $t$ is a task for which the failing function $\Failing(t)$ yields $\True$.
    If a task is failing, we can prove that it accepts no more user input and has no value\citeprop{12}.
    That is, $\Inputs(t) = \nothing$ and $\Value(t,\sigma) = \bot$.
    We also know that, if $\Failing(t)$ holds, it stays failing,
    simply because no input can rewrite $t$ to another $t'$\citeprop{1}.

  \item[Stuck]
    A stuck task $t$ is a task that does not fail, does not accept user input, and does not have a value.
    That is, $\Failing(t) = \False$, $\Inputs(t) = \nothing$, and $\Value(t,\sigma) = \bot$.
    An example of a stuck task is $\View^k 42 \Step \lambda. \Fail$.
    Similar to failing tasks, stuck tasks will always remain stuck because no more user interaction is possible and, by assumption, it is already fully fixated\citeprop{3}.

  \item[Steady]
    If a task $t$ has a value but it cannot handle any input, we call it \emph{steady}.
    An example of a steady task is $\View^k 42$.
    Because this task accepts no more user input, its value will always remain equal to $42$.
    So steady tasks will stay steady\citeprop{2a}.

  \item[Unsteady]
    If a task $t$ has a value and it can handle input, we call it \emph{unsteady}.
    An example of an unsteady task is $\Update^k 42$.
    This task still accepts user input, and thus its value can keep on changing.
    Even though $\Update^k 42$ and $\View^k 42$ yield the same value,
    they should not be equivalent, since one's value can be changed and the other's cannot.
    Unsteady tasks will never terminate.
    They will always remain unsteady with the same input space, only its value may change\citeprop{2b}.
    We do say, however, that both steady and unsteady tasks are \emph{finished}.

  \item[Running]
    The last condition is when a task $t$ does not fail, does not have a value, but still accepts user input.
    Because there is still user interaction possible, it may be the case that, with the right input,
    it transitions to the stuck, steady or unsteady conditions.
    The simplest example of a running task is the empty editor $\Enter^k \beta$,
    which becomes an unsteady task once it receives a valid input event.
    Although running tasks can keep running forever,
    we at least can prove that they will not fail\citeprop{4}.

\end{description}

These five conditions are proved to be mutually exclusive\citeprop{5}.
\cref{tab:task-conditions} summarises all conditions
and shows some more example tasks for each condition.
The possible transitions for tasks from one condition to another are drawn in \cref{fig:task-conditions}.
Moreover, this diagram is proved to be complete:
only those transitions shown are possible \cite{conf/sfp/KlijnsmaS22}.
% for any two task conditions $C$ and $C'$ in \cref{fig:task-conditions},
% there is a transition from $C$ to $C'$
% if and only if there exist two tasks $t \in C$ and $t' \in C'$ such that
% $t,\sigma \interact{\iota} t',\sigma'$ for some input $\iota \in \Inputs(t)$ and states $\sigma$ and $\sigma'$\citeprop{6}.

\begin{table}
  \begin{tabular}{CCClL}
  \toprule
  \Failing   & \Inputs    & \Value     & \alert{Condition} & \alert{Examples}                                                                                                             \\
  \midrule
  \checkmark & -          & -          & Failing           & \Fail, \,\, \Fail \Pair \Fail, \,\, (\lambda x.x) \Trans \Fail, \,\, \Fail \Step \lambda x.\View x \\
  -          & -          & -          & Stuck             & \View 2 \Step \lambda x.\Fail, \,\, \View 2 \Pair \Fail, \,\, \Fail \Pair \View 2                                            \\
  -          & -          & \checkmark & Steady            & \View 2, \,\, \View 2 \Pair \View 3, \,\, \View \tuple{2, 3}                          \\
  -          & \checkmark & \checkmark & Unsteady          & \Update 2, \,\, \Update 2 \Pair \Update 3, \,\, \Update \tuple{2, 3}                                                         \\
  -          & \checkmark & -          & Running           & \Enter \Int \Pair \Fail, \,\, \Update 2 \Step \lambda x.\Fail                                                                \\
  \bottomrule
\end{tabular}
\\
{\small
  Checkmarks for $\Failing$, $\Value$ and $\Inputs$ for a task $t$ and a state $\sigma$ indicate that\\
  $\Failing(t) = \True$,
  $\Value(t, \sigma) = v$ for $v \neq \bot$, and
  $\Inputs(t) \neq \nothing$ respectively.
}

  \caption{Conditions for tasks}
  \label{tab:task-conditions}
\end{table}

\begin{figure}
  \begin{tikzpicture}[x=1pc,y=1pc,shorten >=1pt,minimum size=3.5pc,node distance={0.33pc and 3pc},>={Triangle},draw=black,thick]
  \node[circle,draw,very thick] (running)                            {\emph{Running}};
  \node[circle,draw,very thick] (failing)   [above left=of running]  {\emph{Failing}};
  \node[circle,draw,very thick] (stuck)     [above right=of running] {\emph{Stuck}};
  \node[circle,draw,very thick] (finished)  [below left=of running]  {\emph{Steady}};
  \node[circle,draw,very thick] (recurring) [below right=of running] {\emph{Unsteady}};
  %
  % \draw[thick,->] (running) -- (failing);
  \draw[->] (running) -- (finished);
  \draw[->] (running) -- (recurring);
  \draw[->] (running) -- (stuck);
  \draw[->] (recurring) edge [out=423,in=387,looseness=6]  (recurring);
  \draw[->] (running) edge [out=108,in=72,looseness=6]  (running);
\end{tikzpicture}

  \caption{Possible task conditions and their transitions, where a transition is caused by user interaction ($\interact{\iota}$) for some input $\iota$.}
  \label{fig:task-conditions}
\end{figure}


\subsubsection{Impact of task pools}

Next, we discuss how our theory of task equivalence changes after the addition of task pools.
It is always possible to initialize tasks in pools.
As shown in \cref{fig:observations-dynamic},
the inputs observation $\Inputs$ on a task pool $\PoolText_{t_0}^k [\many{n}]$ will contain the input $k\Send\Init$
regardless of the tasks in the pool.
Also, by definition of $\Value$ in \cref{fig:observations-dynamic},
task pools always have a value.
This value is initially the empty list
and can change at runtime depending on finished tasks in the pool.
We will argue that this has no consequence for the diagram in \cref{fig:task-conditions}.
Let us first consider an example.

\begin{example}[Integer task pool]
  \label{exm:integer-task-pool}
  Let us again revisit our simple task pool to enter integers:
  \begin{TASK}
    let t = >&<$_{\Enter^\epsilon \Int}^{k_0}$ [update$^{k_1}$ 42]
  \end{TASK}

  By definition of $\Value$, it has an observable value, namely the singleton list $[42]$.
  By definition of $\Inputs$, it has multiple possible inputs,
  namely inputs the set $\set{k_0\Send\Init, k_0\Send\Kill1, k_1\Send x \Mid x\in\Int}$.
  This means we can change the value in the editor keyed with $k_1$
  which will change the value of the whole task pool.
  Next to this, we can also kill or initialize tasks in the pool.
  Because $t$ has observable inputs and an observable value that could change,
  it is in the unsteady condition.

  Say, we send $k_0\,\Send\,\Init$ to $t$.
  Then, $t$ will be rewritten to:
  \begin{TASK}
    >&<$_{\Enter^\epsilon \Int}^{k_0}$ [update$^{k_1}$ 42, enter$^{k_2}$ Int].
  \end{TASK}
  The task $\Enter^{k_2}\Int$ does not have a value,
  but the value of the pool itself will not change,
  it stays $[42]$.
  We can, however, still change the value of the pool by sending
  $k_1\Send37$ (resulting in $[37]$) or $k_2\Send43$ (resulting in $[42,43]$).

  We see that $t$ still has a value,
  but some inputs can change this value.
  Therefore it is still in the unsteady condition.
  Note, however, that the input space can change:
  because of dynamically spooled or killed tasks it grows or shrinks.
  % Observation $\Inputs$ will faithfully return
  % $\set{k_0\Send\Init, k_0\Send\Kill1, k_0\Send\Kill2, k_1\Send x, k_2\Send x \Mid x\in\Int}$.
\hfill$\therefore$\end{example}


\subsubsection{Task properties}
\label{sub:task-properties}

In previous work we proved several properties on tasks \cite{conf/sfp/KlijnsmaS22,Steenvoorden22}:
\begin{enumerate}
  \item[(1)] failing tasks stay failing
  \item[(2a)] steady tasks stay steady
  \item[(2b)] unsteady tasks stay unsteady
  \item[(2c)] finished tasks retain their input space
  \item[(3)] stuck tasks stay stuck
  \item[(4)] running tasks will not fail
  \item[(5)] conditions in \cref{fig:task-conditions} are exclusive
  \item[(6)] transitions in \cref{fig:task-conditions} are complete
  % \item[(12)] failing tasks have no inputs and no value
\end{enumerate}
Most of these still hold after the addition of task pools to the language.
This reinforces our belief that the given definitions and semantics for task pools are the right choice.
As pools cannot be failing, steady or stuck by construction,
proving (1), (2a), and (3) is trivial.
Stepping rules are unaltered, therefore (4) still holds.
As we have seen in \cref{exm:integer-task-pool},
task pools are unsteady but the input space can change.
Therefore (2c) does not hold anymore,
but this is not used in the proofs of the other properties.
Notably (5) holds by construction and (6) follows from all other properties except (2c).

This means the only property from this list that needs new proof is (2b).
To do this, we first need to introduce the notion of \emph{stepless tasks}.

\begin{definition}[Stepless task]
  \label{def:stepless-task}
  We call tasks \emph{stepless} when they consist of
  a valued editor,
  pairs of stepless tasks,
  transforms of stepless tasks,
  or task pools of normalized tasks.
  That is, they confirm to the grammar presented in \cref{fig:stepless-tasks-grammar}.
  Note that stepless tasks are a subset of normalized tasks.
\hfill$\therefore$\end{definition}

\begin{figure}
  \GSteplessTaskscompact
  \caption{Stepless tasks}
  \label{fig:stepless-tasks-grammar}
\end{figure}

\begin{proposition}[Valued tasks are stepless]
  \label{prp:valued-means-stepless}
  For all well-typed normalized tasks $n$ and states $\sigma$,
  that is $\Gamma,\Sigma \infers n:\Task\tau$ and $\Gamma,\Sigma \infers \sigma$,
  we have that
    if $\Value(n,\sigma) = v$,
    then $n$ is stepless.
\end{proposition}
\begin{proof}
  We know that $\Value(n,\sigma) = v : \beta$.
  When taking the definition of $\Value$ into account (as given in \cref{fig:observation-value} and \cref{fig:observations-dynamic}),
  we see that $n$ can only be in one of five shapes:
  \begin{itemize}
    \item a valued editor $d$ (that is $d \neq \Enter\beta$);
    \item a pair $n_1 \Pair n_2$, for which $\Value(n_1,\sigma) = v_1$ and $\Value(n_2,\sigma) = v_2$;
    \item a transform $e_1 \Trans n_2$, for which $\Value(n_2,\sigma) = v_2$;
    \item a lift $\Lift v$;
    \item a pool $\PoolText_{t_0} [\many{n}]$.
  \end{itemize}
  Especially, step ($n_1 \Step e_2$) and fail ($\Fail$) are ruled out, because they have no value.
  Choice ($n_1 \Choose n_2$) is ruled out
  because the only way that this combinator can survive normalization is for both $n_1$ and $n_2$ to not have a value,
  but we consider normalized task by assumption.
  Now, by induction, we have that $n$ is stepless and meets the grammar in \cref{fig:stepless-tasks-grammar}.
\end{proof}

Note that inside task pools we admit normalized subtasks instead of stepless subtasks.
This is because our value observation ignores subtasks for which $\Value(n_j,\sigma) = \bot$
and will result in a list of values anyway.

As stepless tasks do not contain steps ($n_1 \Step e_2$) or choices ($n_1 \Choose n_2$),
their \emph{outer shape} cannot be altered at runtime.
That is, although new tasks can be spooled up \emph{inside} task pools,
there is no way for end-users to discard the current task
and step to a new one.
Therefore, stepless tasks stay stepless and the value observation will not result in $\bot$.

\begin{corollary}[Stepless tasks stay stepless]
  \label{cor:stepless-stays-stepless}
  For all well-typed normalized tasks $n$ and states $\sigma$,
  % that is $\Gamma,\Sigma \infers n:\Task\tau$ and $\Gamma,\Sigma \infers \sigma$,
  we have that
    if $n$ is stepless,
    their shape cannot be altered at runtime
    and will stay stepless.
\end{corollary}

Using this observation, we can prove property~(2b).
\begin{proposition}[Unsteady tasks stay unsteady]
  For all unsteady tasks $t$,
  we have that for all inputs $\iota \in \Inputs(t)$:
    if $t,\sigma \interact{\iota} t',\sigma'$
    then again $t'$ is an unsteady task.
\end{proposition}
\begin{proof}
  By definition of the unsteady condition, $t$ has a value,
  that is $\Value(t,\sigma) = v \neq \bot$.
  By \cref{prp:valued-means-stepless}, we know this means that $t$ is stepless.
  Using \cref{cor:stepless-stays-stepless}, we know $t$ stays stepless
  and therefore will keep having a value.
  In particular, it is not possible for any task $t$ to move from the unsteady to the running condition.
\end{proof}

Summarizing, we see that after adding task pools to the \TOPHAT\ language,
its metatheoretical properties stay largely unaltered.
In particular, we can still reason about task equivalence
and transformation laws from previous work still hold.