\newmacro{citeprop}[1]{[Prop.~#1]} % {\cite[Prop.~#1]{conf/sfp/KlijnsmaS22}}

\subsection{Task equivalence}

We do this by categorising tasks based on the observations we can make on them.
The observations we use, are if the task is failing ($\Failing$), if the task can receive any inputs ($\Inputs$), and if the task has a value ($\Value$).
Based on these three observations, a (fully fixated) task can be in one of five \emph{conditions} which we will summarise below.
The mentioned properties are proved in \citet{conf/sfp/KlijnsmaS22}.
We refer to the propositions \todo{}.

\begin{description}

  \item[Failing]
    A failing task $t$ is a task for which the failing function $\Failing(t)$ yields true.
    If a task is failing, we can prove that it accepts no more use input and has no value \citeprop{12}.
    That is, $\Inputs(t) = \empty$ and $\Value(t,\sigma) = \bot$.
    We also know that, if $\Failing(t) = \True$, it stays failing \citeprop{1},
    simply because there is no input which can rewrite $t$ to another $t'$.
    % We have proven this property in \todo{} in the appendix.

  \item[Stuck]
    A stuck task $t$ is a task which does not fail, does not accept user input, and does not have a value.
    That is, $\Failing(t) = \False$, $\Inputs(t) = \empty$, and $\Value(t,\sigma) = \bot$.
    Similar to failing tasks, stuck tasks will always remain stuck because no more user interaction is possible and, by assumption, it is already fully fixated \citeprop{3}.

  \item[Steady]
    If a task $t$ has a value but it cannot handle any input, we call it \emph{steady}.
    An example of a steady task is $\View^k 42$.
    Because this task accepts no more user input, its value will always remain equal to $42$.

  \item[Unsteady]
    If a task $t$ has a value and it can handle input, we call it \emph{unsteady}.
    An example of an unsteady task is $\Update^k 42$.
    This task still accepts user input, and thus its value can keep on changing.
    Even though this example and $\View^k 42$ yield the same value,
    they should not be equivalent, since one's value can be changed and the other one's value cannot.
    If a task $t$ is unsteady, it will never terminate: an unsteady task will always remain unsteady with the same input space, only its value may change \citeprop{2}.

  \item[Running]
    The last condition is when a task $t$ does not fail, does not have a value, but still accepts user input.
    Because there is still user interaction possible, it may be the case that with the right input, it transitions to one of the previously described task conditions.
    The simplest example of a running task is the empty editor $\Enter^k \beta$, which becomes an unsteady task once it receives a valid input event.
    Although running tasks can keep running forever,
    we at least can prove that they will not fail \citeprop{4}.

\end{description}

These conditions are mutually exclusive \citeprop{5}.









Propositions from \cite{Steenvoorden22} and \cite{Klijnsma2020} that still hold are the following:

\begin{itemize}
  \item Valued tasks are static
  \item Steady tasks stay steady
\end{itemize}

However, because of the dynamic nature of task pools, the following propositions need alteration:

\begin{itemize}
  \item Static tasks stay static
  \item Finished tasks stay finished
\end{itemize}

This is because task pools can always receive the initialise input ($\Init$),
so tasks containing a task pool will never get to a steady state!

We will discuss the impact on equational reasoning on tasks more thoroughly...