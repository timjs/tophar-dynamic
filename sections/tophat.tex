% !TEX root=../main.tex

\section{TopHat}
\label{sec:tophat}

Before delving into the details of \DYNTOPHAT, we take some time to recall \TOPHAT.
The sections below present the language, its syntax and semantics, and observations over tasks.

\subsection{Language}

We describe the \TOPHAT\ language as presented in \citet{Steenvoorden22}.
This slightly deviates from presentations in earlier work \cite{conf/ppdp/SteenvoordenNK19,conf/ifl/NausSK19,conf/sfp/NausS20}.

\begin{figure}[b]
  \usemacro{G-Tasks-compact}
  \caption{Task grammar}
  \label{fig:task-grammar}
\end{figure}

The \TOPHAT\ language consists of two layers: the expression layer and the task layer.
The expression layer is a \STLC\ with primitive types, pairs, lists, enumerations, and references.
We use pattern matching to project the fields of pairs.
As the host language is fairly standard, we omit its details here
and refer to \citet{Steenvoorden22} for a thorough description.
The grammar of the task layer is given in \cref{fig:task-grammar}.
Tasks can either be editors or combinators.
We will discuss both in the following subsections.

\subsubsection{Editors}

Editors are the interaction points of task programs.
They are an abstraction over widgets or form input fields.
Editors are type-safe:
they only accept values of their specified type.
The type of an editor can also be used to generically render a user interface.
For example,
an editor of type $\Bool$ could be rendered as a check box,
while end-users could interact with an editor of type $\Location$ by putting a pin on a map.

Editors can initially be empty, or \emph{unvalued} ($\Enter^\nu \beta$).
After sending a value $b$ to an editor, the editor is now \emph{valued} ($\Update^\nu b$).
The value in an editor can be changed by sending new values.
However, they cannot be cleared.
Editors can also be declared as \emph{read-only} ($\View^\nu b$).
This way they show information to end-users that cannot be changed.
Editors are tagged by a name $\nu$.
Names are used to distinguish editors when sending input events (see \cref{sub:inputs} below).
Possible states editors can be in are depicted in \cref{fig:editor-state}.

\begin{figure}
  \input{figures/editors}
  \caption{Possible states of editors and their transitions}
  \label{fig:editor-state}
\end{figure}

Shared editors $\Change^\nu h$ watch over heap locations $h$,
similar to \ML-style references.
Changing the value of shared editors changes the value in the heap location.
Multiple shared editors can watch the same heap location.
As plain editors, shared editors can also be read-only ($\Watch^\nu h$),
which means their displayed value will only change when the referenced heap location changes,
but end-users cannot send values to update the heap location.


\subsubsection{Combinators}

Combinators unite smaller tasks into bigger ones.
The parallel-and combinator ($t_1 \Pair t_2$) for example,
combines two tasks so that both can be worked on concurrently.
The result thereof is a pair containing both the results of $t_1$ and $t_2$.
The parallel-or combinator ($t_1 \Choose t_2$) on the other hand,
results in either the result of $t_1$ or $t_2$,
depending on which task finished first.
To map a function over a task's value,
one can use the transform task ($v_1 \Trans t_2$).

The most important combinator is the step combinator.
The task $t_1 \Step v_2$ decides when to transition from task $t_1$
to the task calculated by its continuation $v_2$.
The continuation should be a function, taking an ordinary value resulting in a task to continue with.
The typing rule of this combinator is shown in \cref{fig:typing-rules}.
The failing task $\Fail$ is used to signal a task that cannot be stepped to.
Also, it is the identity of parallel-or tasks.

Furthermore, one can create shared heap locations ($\Share v$) and assign to them ($h_1 \Assign v_2$).
The lift task ($\Lift v$) is used to lift a value into the task domain.

\begin{example}[Vending machine]
  \label{exm:vending-base}
  The following example demonstrates the use of editors and the step combinator.
  We have a vending machine that dispenses a biscuit for one coin and a chocolate bar for two coins.
  % \begin{TASK}
  %   let vend : Task Snack = enter Int >>? do n. if n == 1 then view Biscuit
  %     else if n == 2 then view Chocolate else fail
  % \end{TASK}
  \begin{TASK}[emph={payment}]
    type Snack = [Biscuit, Chocolate]
    let vend : Task Snack = enter Int >>= do payment.
      case payment of
        1 |-> view Biscuit
        2 |-> view Chocolate
        _ |-> fail
  \end{TASK}
  The editor $\Enter \Int$ asks the user to enter an amount of money.
  This editor stands for a coin slot in a real machine that freely accepts and returns coins.
  % There is a continue button on the machine, which sends a continue event to the select combinator ($\Select$).
  % This is a derived combinator acting similar to the previously mentioned step combinator.
  % The button is initially disabled because the editor has no value.
  When the user has inserted exactly 1 or 2 coins, the step can be made.
  % When the user presses the continue button,
  The machine dispenses either a biscuit or a chocolate bar, depending on the amount of money.
  Other cases result in the failure task $\Fail$, which signals a path that cannot be followed.
  In those cases, the vending machine stays in the state to accept and return money.
  Snacks are modeled using a custom type.
\end{example}


\subsubsection{Inputs}
\label{sub:inputs}

Editors are the interaction leaves of task-oriented programs.
They are the only way users can interact with the system.
Initially, editors are unnamed ($\epsilon$).
After the system prepared them for user interaction, a process we call normalisation,
each editor is keyed with a \emph{unique} key $k$.
Sending the input $k \Send b$ to the system will change the value inside the editor with key $k$ to hold value $b$.
\cref{fig:input-grammar} contains the grammar of inputs and actions accepted by a \TOPHAT\ system.
In the remaining part of this article,
we omit names and keys when they are not relevant to the example.

\begin{figure}
  \usemacro{G-Inputs-compact}
  \caption{Input grammar}
  \label{fig:input-grammar}
\end{figure}


\subsection{Semantics}

\subsubsection{Typing}

\begin{figure}[b]
  \centering
  \usemacro{G-Types-compact}
  \caption{Typing grammar}
  \label{fig:typing-grammar}
\end{figure}

\cref{fig:typing-grammar} shows the grammar of types.
Typing rules of a selected number of tasks can be found in \cref{fig:typing-rules}.
Note that editors and references can only contain values of basic types $\beta$.
In particular, editors and references cannot contain functions or tasks themselves.
The reason for this restriction is twofold.
First, prohibiting references to functions keeps our host language total.
% This is necessary for the symbolic execution to not enter an infinite loop.
Second, restricting editors to only contain basic values
guarantees one can present these values to end-users (print them)
and receive new inputs from end-users (parse them).
Arbitrary tasks and functions do not have this property.

\begin{figure}
  % \small
  \begin{mathpar}
    \boxed{\RelationT}    \\
    % \userule{T-Enter}     \quad
    \userule{T-Update}    \quad
    \userule{T-Share}     \\
    % \userule{T-View}      \\
    % \userule{T-Change}    \quad
    % \userule{T-Watch}     \\
    % \userule{T-Trans}     \\
    \userule{T-Pair}      \quad
    % \userule{T-Lift}      \\
    % \userule{T-Choose}    \quad
    \userule{T-Fail}      \\
    \userule{T-Step}      \\
    % \userule{T-Assign}
  \end{mathpar}
  \caption{Selected typing rules}
  \label{fig:typing-rules}
\end{figure}


\subsubsection{Semantic relations}

\TOPHAT's dynamic semantics consist of three layers: the host layer, the internal layer, and the external layer.
Each contains one or two semantic relations, as depicted in \cref{fig:semantic-layers}.

\begin{figure}
  \usemacro{P-Semantic-Layers}
  \caption{Semantic functions and their relation
    (read $x \bullet\!\!\raisebox{-0.1ex}{\mbox{–}}\ y$ as `$x$ makes use of $y$')}
  \label{fig:semantic-layers}
\end{figure}

\begin{description}
  \item[Host layer]
    The host layer is responsible for evaluating expressions in our host language,
    the \STLC\ with basic types and references.
    This layer only contains one big-step evaluation relation ($\evaluate$).
  \item[Internal layer]
    The internal layer only operates on tasks.
    It prepares tasks so they can be presented to end-users to interact with.
    This layer contains the big-step normalization ($\normalise$) and fixation ($\fixate$) relations.
  \item[External layer]
    The external layer takes inputs from end-users and rewrites tasks based on this input.
    This layer contains two small-step relations: handling ($\handle{\iota}$) and interaction ($\interact{\iota}$).
    Both relations are labeled with inputs $\iota$.
\end{description}

\begin{figure}
  \begin{mathpar}
    \boxed{\RelationN} \\
    \userule{N-Assign} \Quad
    \userule{N-Name} \\
    \userule{N-Pair} \\
  \end{mathpar}
  \caption{Selected normalisation rules}
  \label{fig:semantics-normalisation}
\end{figure}

In \cref{fig:semantics-normalisation} we give some examples of normalisation rules in \TOPHAT.
The normalisation relation reads $\RelationN$, which means that a task $t$ in state $\sigma$
is normalised to a normalised task $n'$ in state $\sigma'$ while keeping track of possible dirty references in $\delta'$.
Normalised tasks are a subset of tasks.
Their grammar can be found in \cref{fig:task-grammar}.

A heap location is added to the dirty reference set $\delta$ when it is modified.
An example of this is assigning a value, as seen in \seerule{N-Assign}.
A non-empty set of dirty references signals the fixation rules ($\RelationF$) to fire
to resolve any dependencies between shared heap locations.

Rule \refrule{N-Name} generates a fresh and unique key $k$ to annotate any editor.
\refrule{N-Pair} exemplifies a rule where each element of a combinator is normalised by threading state $\sigma$,
and the set of dirty references is calculated by taking the union of the generated dirty sets.

\begin{figure}
  \begin{mathpar}
    \boxed{\RelationH} \\
    \userule{H-Enter} \\
    \userule{H-Pair} \\
  \end{mathpar}
  \caption{Selected handling rules}
  \label{fig:semantics-handling}
\end{figure}

On the external layer, we show some example handling rules in \cref{fig:semantics-handling}.
Sending a basic value $b'$ to an unvalued editor will fill the editor with that value (\seerule{H-Enter}).
Note that the key in the input event should match the key of the editor.
Also, the type of the basic value should match the type the editor expects.
Sending input events to a parallel-and will pass it on to its components
and merge the resulting tasks and dirty sets (\seerule{H-Pair}).


\subsubsection{Observations}
\label{sec:observations}

Next to the five semantic relations, \TOPHAT\ has the notion of \emph{observations} on tasks.
Observations are used to decide which semantic rule will fire,
and are also a central aspect of task equivalence,
as we will see in \cref{sec:equivalence}.
These observations include, amongst others,
if a task is failing or not ($\Failing$),
the set of possible inputs a task can handle ($\Inputs$),
and the value of a task ($\Value$).
\cref{fig:semantic-layers} shows which layers of the semantics use which observations.
Next, we describe each of these three observations in more detail.

\begin{figure}
  \usemacro{O-Failing}
  \caption{Failing observation}
  \label{fig:observation-failing}
\end{figure}

Not just $\Fail$ can fail, tasks with failing subtasks can also fail.
For example, the pairing $\Fail \Pair \Fail$ is also a failing task.
In some way, it is equivalent to $\Fail$.
To capture what tasks are failing we define the failing observation $\Failing$ in \cref{fig:observation-failing}.
Failing is especially useful when used in combination with the step combinator,
as we have seen in \cref{exm:vending-base}.
Note that if only one side fails in a parallel-and,
for example with $\Fail \Pair t$ for $t$ a non-failing task,
then $\Fail \Pair t$ can still interact with end-users if $t$ can.
Therefore the whole task should not fail if only one side fails.

\begin{figure}[b]
  \usemacro{O-Inputs}
  \caption{Inputs observation}
  \label{fig:observation-inputs}
\end{figure}

\begin{figure}[b]
  \usemacro{O-Value}
  \caption{Value observation}
  \label{fig:observation-value}
\end{figure}

Once a TopHat program is normalised, it is possible to send input to it.
It is useful to know what inputs a given task accepts.
The observation function $\Inputs$ returns the set of input events that are currently possible for a given task.
We consider an input event $\iota$ a valid input event for the task $t$ iff $t \in \Inputs(t)$.
This observation is proved sound and complete with respect to the handling and interaction relations \cite{Steenvoorden22}.
The definition of $\Inputs$ is given in \cref{fig:observation-inputs}.
For the three read-write editors, $\Inputs$ returns all input events $k \Send b$ where $b$ is of the correct type.
For task combinators, $\Inputs$ is defined recursively.
For all other tasks, $\Inputs$ returns the empty set.
$\Inputs$ is only defined on normalised tasks, as those tasks are ready to receive user input.

To be able to normalise tasks, we need to have a way to determine the value of them.
For this, we introduce the task observation $\Value$.
Given a task $t:\Task\tau$ and the current state $\sigma$ ,
this function returns the task's value $v$ of type $\tau$.
It is also possible that a task's value is undefined, in which case we write $\Value(t,\sigma) = \bot$.
The definition of $\Value$ is given in \cref{fig:observation-value}.