% !TEX root=../main.tex
%
%The abstract should briefly summarize the contents of the paper in
%150--250 words.
%
% Context
\TOPHAT\ is a mathematically formalised language for Task-Oriented Programming (\TOP).
It allows developers to specify workflows and business processes in a formal language,
reason about their equality
and use symbolic execution to verify their correctness.
% Inquiry
\TOPHAT\ can run a specification supporting collaborators during the execution of a workflow.
However, it can only do so for a statically specified amount of work.
That is, the number of tasks running in parallel is always predefined by the developer.
In contrast, other \TOP\ engines like \ITASKS\ and \MTASKS\ act like an operating system,
starting and stopping tasks at will.

% Approach
To capture this dynamic nature of workflow systems,
we introduce \DYNTOPHAT:
a moderate extension to the \TOPHAT\ calculus which allows end users to initialise and kill tasks at runtime.
% Knowledge
Although this is a restricted version of the dynamic tasks lists found in \ITASKS,
where the system itself can initialise new tasks,
we show that all common use cases of this feature are still expressible in \DYNTOPHAT.
Also, our proposed solution does not compromise the formal reasoning properties of \TOPHAT.
% Grounding
\TOPHAT's metatheory is formalised in the dependently typed programming language \IDRIS\
and it's symbolic execution engine is implemented in \HASKELL.
% Importance


% Context: What is the broad context of the work? What is the importance of the general research area?
% Inquiry: What problem or question does the paper address? How has this problem or question been addressed by others (if at all)?
% Approach: What was done that unveiled new knowledge?
% Knowledge: What new facts were uncovered? If the research was not results oriented, what new capabilities are enabled by the work?
% Grounding: What argument, feasibility proof, artifacts, or results and evaluation support this work?
% Importance: Why does this work matter?