% !TEX root=../main.tex

\section{Related Work}
\label{sec:relatedwork}

The development of workflow software has shaped computer science and the IT industry almost from its inception onwards.
The earliest example of this is the development of the object-oriented (OO) programming paradigm.
OO offers features such as abstractions in terms of objects, inheritance, encapsulation and event-driven programming, that are tailored toward modeling and programming business processes.

OO also comes with its own set of challenges.
In the late 90s, the Unified Process was presented to overcome some of the struggles developers were having~\cite{DBLP:books/daglib/0000196}.

Since then, more powerful abstractions have been developed to deal with the automation of workflow processes.
Below, we list a few of these directions and key works associated with them.

\paragraph{Workflow modeling}

Research on workflow modeling tends to focus on describing control flow, mostly in a \textit{boxes and arrows} type of way.
The Workflow patterns by \citet{journals/dpd/AalstHKB03} are a prime example of this. 
Common patterns in workflow software, akin to the design patterns in software engineering, are described as control flow graphs.
Workflow nets are a variation on workflow patterns, where Petri nets are used instead of control flow graphs~\cite{journals/infsof/LassenA09,journals/jcsc/Aalst98}.
YAWL builds upon this, extending the workflow nets in such a way that they are directly executable~\cite{journals/is/AalstH05}.
YAWL systems are programmed visually.

BPEL \todo{write me!}

\paragraph{Process algebra}

Process algebras model concurrency.
They allow for the high-level description of interaction, communication and synchronization between different processes.
Well known examples of process algebras are CSP~\cite{books/ph/Hoare85}, CCS~\cite{books/ph/Milner89}, and $\pi$~\cite{DBLP:books/daglib/0098267}.
There are two main differences between \TOP\ and process algebras.
Firstly, process algebras focus on modeling the input/output behavior of processes, by explicitly stating which actions are sent and received.
\TOP\ on the other hand is aimed at modeling collaboration patterns, with the explicit goal of not having to specify how exactly subtasks communicate.
The second difference concerns internal communication.
Two forms of communication between tasks exist: Passing values to continuations and sharing data.
This is different from communication in process algebras, which is based on message-passing.

\paragraph{Reactive programming}
HipHop
Esterel
Functional reactive programming

\paragraph{Task-oriented programming}

\TOP\ has been studied for almost twenty years now.
\citet{conf/icfp/PlasmeijerAK07} present the foundations of having a work-flow specification that is embedded in a functional language and is actually executable.
Later work presents a full implementation of \TOP\ in the form of \ITASKS, a framework embedded in \CLEAN\ to develop web-based interactive workflow programs~\cite{conf/ppdp/PlasmeijerLMAK12}.

Since then, a broad range of tools and extensions have been presented in various works, such as a distributed version~\cite{conf/ifl/OortgieseGAP17}, a task visualization system~\cite{conf/sfp/StutterheimPA14,conf/cefp/StutterheimAP15}, tasks for embedded systems~\cite{conf/cgo/KoopmanLP18,conf/ifl/LubbersKP18,conf/mipro/LubbersKP19,conf/ifl/LubbersKP19}, just to name a few.

Even though \ITASKS\ and its many extensions are implemented as DSL in a functional language, it lacks a formal foundation, making it harder to reason about \TOP\ programs.
To facilitate reasoning over program correctness, asserting program properties and performing symbolic execution, a more formal implementation of \TOP\ is needed.

This is why \TOPHAT\ was developed~\cite{conf/ppdp/SteenvoordenNK19}.
The \TOPHAT\ language, as introduced in Section~\ref{sec:tophat}, has a formal syntax and semantics.
This allowed for the development of a symbolic execution semantics~\cite{conf/ifl/NausSK19}, a user feedback system~\cite{conf/sfp/NausS20} and equivalence reasoning over programs~\cite{conf/sfp/KlijnsmaS22}.
Details on these systems have been given above in Section~\ref{sec:properties}.