% !TEX root=../main.tex

\section{Related Work}
\label{sec:relatedwork}

The development of workflow software has shaped computer science and the IT industry almost from its inception onwards.
The earliest example of this is the development of the object-oriented (OO) programming paradigm.
OO offers features such as abstractions in terms of objects, inheritance, encapsulation and event-driven programming, that are tailored toward modeling and programming business processes.\todo{cite needed!}

OO also comes with its own set of challenges.
In the late 90s, the Unified Process was presented to overcome some of the struggles developers were having.

Since then, more powerful abstractions have been developed to deal with the automation of workflow processes.
Below, we list a few of these directions and key works associated with them.

\paragraph{Workflow modelling}
Workflow patterns~\cite{journals/dpd/AalstHKB03}
workflow nets~\cite{journals/infsof/LassenA09,journals/jcsc/Aalst98}
YAWL~\cite{journals/is/AalstH05}

\paragraph{Process calculi}
CSP~\cite{books/ph/Hoare85}
CCS~\cite{books/ph/Milner89}
Pi~\cite{DBLP:books/daglib/0098267}

\paragraph{Reactive programming}
HipHop
Esterel
Functional reactive programming

\paragraph{Task-oriented programming}

TOP has been studied for a long time:

executable specifications of interactive work flow systems for the web~\cite{conf/icfp/PlasmeijerAK07}
TOP in a pure functional language~\cite{conf/ppdp/PlasmeijerLMAK12}
distributed top~\cite{conf/ifl/OortgieseGAP17}

These implementations lack a formal foundation, making it harder to reason about stuff like correctness, equivalence, valuation, etc.

tophat~\cite{conf/ppdp/SteenvoordenNK19}

This formal foundation has lead to development of formal methods in TOP:

symbolic tophat~\cite{conf/ifl/NausSK19}
top with feedback~\cite{conf/sfp/NausS20}
Equivalence of programs~\cite{conf/sfp/KlijnsmaS22}