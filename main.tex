%%
%% This is file `sample-manuscript.tex',
%% generated with the docstrip utility.
%%
%% The original source files were:
%%
%% samples.dtx  (with options: `manuscript')
%% 
%% IMPORTANT NOTICE:
%% 
%% For the copyright see the source file.
%% 
%% Any modified versions of this file must be renamed
%% with new filenames distinct from sample-manuscript.tex.
%% 
%% For distribution of the original source see the terms
%% for copying and modification in the file samples.dtx.
%% 
%% This generated file may be distributed as long as the
%% original source files, as listed above, are part of the
%% same distribution. (The sources need not necessarily be
%% in the same archive or directory.)
%%
%% Commands for TeXCount
%TC:macro \cite [option:text,text]
%TC:macro \citep [option:text,text]
%TC:macro \citet [option:text,text]
%TC:envir table 0 1
%TC:envir table* 0 1
%TC:envir tabular [ignore] word
%TC:envir displaymath 0 word
%TC:envir math 0 word
%TC:envir comment 0 0
%%
%%
%% The first command in your LaTeX source must be the \documentclass command.
%%%% Small single column format, used for CIE, CSUR, DTRAP, JACM, JDIQ, JEA, JERIC, JETC, PACMCGIT, TAAS, TACCESS, TACO, TALG, TALLIP (formerly TALIP), TCPS, TDSCI, TEAC, TECS, TELO, THRI, TIIS, TIOT, TISSEC, TIST, TKDD, TMIS, TOCE, TOCHI, TOCL, TOCS, TOCT, TODAES, TODS, TOIS, TOIT, TOMACS, TOMM (formerly TOMCCAP), TOMPECS, TOMS, TOPC, TOPLAS, TOPS, TOS, TOSEM, TOSN, TQC, TRETS, TSAS, TSC, TSLP, TWEB.
% \documentclass[acmsmall]{acmart}

%%%% Large single column format, used for IMWUT, JOCCH, PACMPL, POMACS, TAP, PACMHCI
% \documentclass[acmlarge,screen]{acmart}

%%%% Large double column format, used for TOG
% \documentclass[acmtog, authorversion]{acmart}

%%%% Generic manuscript mode, required for submission
%%%% and peer review
\documentclass[sigconf,screen,review]{acmart}
%% Fonts used in the template cannot be substituted; margin 
%% adjustments are not allowed.
%%
%% \BibTeX command to typeset BibTeX logo in the docs
\AtBeginDocument{%
  \providecommand\BibTeX{{%
    \normalfont B\kern-0.5em{\scshape i\kern-0.25em b}\kern-0.8em\TeX}}}

%% Rights management information.  This information is sent to you
%% when you complete the rights form.  These commands have SAMPLE
%% values in them; it is your responsibility as an author to replace
%% the commands and values with those provided to you when you
%% complete the rights form.
\setcopyright{acmcopyright}
\copyrightyear{2018}
\acmYear{2018}
\acmDOI{XXXXXXX.XXXXXXX}

%% These commands are for a PROCEEDINGS abstract or paper.
\acmConference[IFL '23]{The 35th Symposium on Implementation and Application of Functional Languages}{August 29 -- 31, 2023}{Braga, Portugal}
%
%  Uncomment \acmBooktitle if th title of the proceedings is different
%  from ``Proceedings of ...''!
%
\acmBooktitle{IFL '18: The 35th Symposium on Implementation and Application of Functional Languages,
 August 29--31, 2023, Braga, Portugal} 
\acmPrice{15.00}
\acmISBN{978-1-4503-XXXX-X/18/06}


%%
%% Submission ID.
%% Use this when submitting an article to a sponsored event. You'll
%% receive a unique submission ID from the organizers
%% of the event, and this ID should be used as the parameter to this command.
%%\acmSubmissionID{123-A56-BU3}

%%
%% For managing citations, it is recommended to use bibliography
%% files in BibTeX format.
%%
%% You can then either use BibTeX with the ACM-Reference-Format style,
%% or BibLaTeX with the acmnumeric or acmauthoryear sytles, that include
%% support for advanced citation of software artefact from the
%% biblatex-software package, also separately available on CTAN.
%%
%% Look at the sample-*-biblatex.tex files for templates showcasing
%% the biblatex styles.
%%

%%
%% The majority of ACM publications use numbered citations and
%% references.  The command \citestyle{authoryear} switches to the
%% "author year" style.
%%
%% If you are preparing content for an event
%% sponsored by ACM SIGGRAPH, you must use the "author year" style of
%% citations and references.
%% Uncommenting
%% the next command will enable that style.
%%\citestyle{acmauthoryear}

%%
%% end of the preamble, start of the body of the document source.
\begin{document}

%%
%% The "title" command has an optional parameter,
%% allowing the author to define a "short title" to be used in page headers.
\title{Formal definition of TOP lists}

%%
%% The "author" command and its associated commands are used to define
%% the authors and their affiliations.
%% Of note is the shared affiliation of the first two authors, and the
%% "authornote" and "authornotemark" commands
%% used to denote shared contribution to the research.
\author{Tim Steenvoorden}
\email{tim.steenvoorden@ou.nl}
\orcid{0002-8436-2054}
\affiliation{%
  \institution{Open University of The Netherlands}
  \city{Heerlen}
  \country{The Netherlands}
}

\author{Nico Naus}
\email{nico.naus@ou.nl}
\orcid{0003-3442-1543}
\affiliation{%
  \institution{Open University of The Netherlands}
  \city{Heerlen}
  \country{The Netherlands}
}

%%
%% By default, the full list of authors will be used in the page
%% headers. Often, this list is too long, and will overlap
%% other information printed in the page headers. This command allows
%% the author to define a more concise list
%% of authors' names for this purpose.
\renewcommand{\shortauthors}{Steenvoorden and Naus}

%%
%% The abstract is a short summary of the work to be presented in the
%% article.
\begin{abstract}

% !TEX root=../main.tex
%
%The abstract should briefly summarize the contents of the paper in
%150--250 words.
%
% Context
\TOPHAT\ is a mathematically formalised language for Task-Oriented Programming (\TOP).
It allows developers to specify workflows and business processes in a formal language,
reason about their equality
and use symbolic execution to verify their correctness.
% Inquiry
\TOPHAT\ can run a specification to support collaborators during the execution of a workflow.
However, it can only do so for a statically specified amount of work.
That is, the number of tasks running in parallel is always predefined by the developer.
In contrast, other \TOP\ engines like \ITASKS\ and \MTASKS\ act like an operating system,
starting and stopping tasks at will.

% Approach
To capture this dynamic nature of workflow systems,
we introduce \DYNTOPHAT:
a moderate extension to the \TOPHAT\ calculus which allows end users to initialise and kill tasks at runtime.
% Knowledge
Although this is a restricted version of the dynamic tasks lists found in \ITASKS,
where the system itself can initialise new tasks,
we show that all common use cases of this feature are still expressible in \DYNTOPHAT.
Also, our proposed solution does not compromise the formal reasoning properties of \TOPHAT.
% Grounding
\TOPHAT's metatheory is formalised in the dependently typed programming language \IDRIS\
and it's symbolic execution engine is implemented in \HASKELL.
% Importance


% Context: What is the broad context of the work? What is the importance of the general research area?
% Inquiry: What problem or question does the paper address? How has this problem or question been addressed by others (if at all)?
% Approach: What was done that unveiled new knowledge?
% Knowledge: What new facts were uncovered? If the research was not results oriented, what new capabilities are enabled by the work?
% Grounding: What argument, feasibility proof, artifacts, or results and evaluation support this work?
% Importance: Why does this work matter?

\end{abstract}

%%
%% The code below is generated by the tool at http://dl.acm.org/ccs.cfm.
%% Please copy and paste the code instead of the example below.
%%
\begin{CCSXML}
<ccs2012>
 <concept>
  <concept_id>10010520.10010553.10010562</concept_id>
  <concept_desc>Computer systems organization~Embedded systems</concept_desc>
  <concept_significance>500</concept_significance>
 </concept>
 <concept>
  <concept_id>10010520.10010575.10010755</concept_id>
  <concept_desc>Computer systems organization~Redundancy</concept_desc>
  <concept_significance>300</concept_significance>
 </concept>
 <concept>
  <concept_id>10010520.10010553.10010554</concept_id>
  <concept_desc>Computer systems organization~Robotics</concept_desc>
  <concept_significance>100</concept_significance>
 </concept>
 <concept>
  <concept_id>10003033.10003083.10003095</concept_id>
  <concept_desc>Networks~Network reliability</concept_desc>
  <concept_significance>100</concept_significance>
 </concept>
</ccs2012>
\end{CCSXML}

\ccsdesc[500]{Computer systems organization~Embedded systems}
\ccsdesc[300]{Computer systems organization~Redundancy}
\ccsdesc{Computer systems organization~Robotics}
\ccsdesc[100]{Networks~Network reliability}

%%
%% Keywords. The author(s) should pick words that accurately describe
%% the work being presented. Separate the keywords with commas.
\keywords{Task oriented programming, formal semantics, symbolic execution, functional programming}

%% A "teaser" image appears between the author and affiliation
%% information and the body of the document, and typically spans the
%% page.

%%
%% This command processes the author and affiliation and title
%% information and builds the first part of the formatted document.
\maketitle

% !TEX root=../main.tex

\section{Introduction}
\label{sec:introduction}

In modern society, it is almost unthinkable to run any organization without the help of software that manages the organization's processes.
From managing customers and patients to monitoring systems and situations, software has become a crucial component.

Traditionally, these systems are developed using tools that offer a rather low level of abstraction.
Standard Object-Oriented approaches allow some superficial domain modelling and best practices like the Unified Process from the late 90's provide developers with some guidance.

Many researchers found that more powerful abstractions are needed to develop these all-too-common systems in a way that results in more robust and reliable systems.
Since then, a lot of work has been published on better describing and implementing business processes, such as the workflow patterns by \citeauthor{journals/dpd/AalstHKB03}, Workflow Nets \cite{journals/jcsc/Aalst98}, and business process calculi like \BPEL\ \cite{bpel}.

Most notable is the work by \citet{conf/ifl/KoopmanPA08} and \citet{conf/ppdp/PlasmeijerLMAK12} on \ITASKS.
They present an abstraction over workflow systems implemented as a \DSL\ in the functional programming language \CLEAN\ \cite{plasmeijer2002clean}.
The idea behind their paradigm is to model collaboration patterns, and to abstract away from things like \GUI, client-server communication, and databases.
Their work proved to be successful, with several extensions and the technology now being used in a spin-off company \cite{com/tss/viia}.

Even though \ITASKS\ is implemented as a \DSL\ in a functional programming language, the paradigm lacks any formal semantics.
This heavily restricts the amount of formal reasoning that can be done on \ITASKS\ programs.

To overcome this downside, \TOPHAT\ was introduced~\cite{conf/ppdp/SteenvoordenNK19}.
Built on the \STLC,
\TOPHAT\ is a \TOP\ implementation that has a complete formal semantics.
Several works leverage this to perform symbolic execution \cite{conf/ifl/NausSK19},
generate next-step hints for end-users \cite{conf/sfp/NausS20},
and reason about equivalence of tasks \cite{conf/sfp/KlijnsmaS22}.
% A complete and thorough description of all of \TOPHAT's features can be found in \citet{Steenvoorden22}.

Compared to the \ITASKS\ system, \TOPHAT\ is more restrictive in managing tasks itself.
\TOPHAT\ programs can only define a static amount of work, whereas \ITASKS\ programs can dynamically spool up new tasks and terminate old ones.
In this paper, we introduce \DYNTOPHAT, an extension to \TOPHAT\ that allows end-users to initialize and kill tasks at runtime.
Key in our approach is that we do not compromise the formal properties when adding this feature.

% \paragraph{Contributions}

Our main contributions are as follows:
\begin{itemize}
  \item
    We extend the basic \TOPHAT\ language with dynamic \emph{task pools},
    allowing end-users to initialize and kill tasks at runtime.
    We call this extension \DYNTOPHAT.
  \item
    We reason that symbolic execution of tasks containing pools can result in infinite simulation.
    However, we show that it is still possible to use symbolic execution to generate next-step hints for end-users
    by altering our algorithm from previous work.
  \item
    We prove that the addition of task pools has a small impact on the metatheoretical properties of \TOPHAT.
    We can still reason about the evolution of tasks at runtime
    and prove when two tasks are equal.
    Also, previously published transformation laws on tasks still hold:
    the \typ{Task} operation on types is a functor but cannot be a monad.
\end{itemize}

% \todo{Say something about validation and proofs here?}

% \paragraph{Structure}

The remainder of this paper is structured as follows.
\cref{sec:tophat} introduces the \TOPHAT\ language as presented in \citet{Steenvoorden22}.
\cref{sec:dyntophat} then goes on to present our dynamic version of \TOPHAT.
% After having introduced all language constructs, we study a bigger example in \cref{sec:example}.
We then study the implications on \TOPHAT's properties in \cref{sec:properties}.
We discuss the impact of task pools on the symbolic execution mechanism and task equivalence.
Related work is discussed in \cref{sec:relatedwork}.
After a short discussion on the design decisions of \DYNTOPHAT\ in \cref{sec:discussion},
we conclude in \cref{sec:conclusion}.

% !TEX root=../main.tex

\section{Conclusion}
\label{sec:conclusion}

We presented the problem that programmers need to specify the number of parallel tasks in \TOPHAT\ during development time
and introduced \emph{task pools} as a solution.
In the resulting language \DYNTOPHAT\ is a moderate extension to \TOPHAT\ with more dynamic properties.
At runtime, end-users can initialise and kill tasks in a pool at will.
We presented the static and dynamic semantics of task pools extending \TOPHAT\ multiple semantic layers.
We were able to do this in such a way that the impact on formal reasoning of \TOPHAT\ programs is minimal.
Although symbolic execution could end in an infinite loop,
we altered our next-step hint generation system so that it can still support end-users during workflow execution.
Also, our semantic extensions were defined in such a way, that equational reasoning on tasks is still possible
and transformation laws proved in earlier work still hold.

This reinforces our belief that the given definitions and semantics for task pools are the right choice.

% !TEX root=../main.tex

\section{Discussion}

\paragraph{Future work}
\label{sec:future-work}

In the future, we would like to investigate the amount of real-world tasks written in \ITASKS\ that could be rewritten by using task pools.
This way, such tasks could benefit from validation by symbolic execution without compromising expressivity.
Also, developers could use the transformation rules from \citet{conf/sfp/KlijnsmaS22} to refactor their programs.

It would be even more interesting, however, when task pools will fall short in these real-world examples.
In that case, we would like to investigate which additional power is needed in \TOPHAT\ to write these programs and reason about them.


\begin{acks}
...
\end{acks}

%% The next two lines define the bibliography style to be used, and
%% the bibliography file.
\bibliographystyle{ACM-Reference-Format}
\bibliography{main}


\end{document}
\endinput
%%
%% End of file `sample-authordraft.tex'.
